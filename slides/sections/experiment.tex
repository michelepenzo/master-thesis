% Introduction section

\section{Experiment}

% ------------------------------------------
\subsection{Experimental design}

\begin{frame}
	\frametitle{\subsecname}
	The experimental design describes the entire flow which is done by every participants of the experiment.\\~\ 
	
	During the experiment some data were collected from ROS\texttt{ topic} with sample rate of $ 10Hz $. Those data are related to cartesian pose, cartesian wrench, joint position and joint velocity.\\~\
	
	The experiment has been divided into three stages for convenience:
	\begin{enumerate}
		\item pre experiment 
		\item experiment 
		\item post experiment 
	\end{enumerate}
\end{frame}

\subsection{Pre experiment phase}

\begin{frame}
	\frametitle{\subsecname}	
	At this stage users were asked to take confidence with both the modalities.\\~\
	
	If the user complete this phase an ascending unique id is assigned to him.\\~\
	
	Every user perform the pre-experiment and experiment phase without seeing other users do the same.
\end{frame}


\subsection{Experimental phase}

\begin{frame}
	\frametitle{\subsecname}
	The experiment was repeated both for \kt{} and \te.\\~\
	
	It consists in two tasks in ascending order of difficulty repeated for three times.	For simplicity the repetitions were called ``trial''. \\~\
	
	The figure on the left is a pick and place with a simple interlocking, while the figure on the right is a difficult interlocking.
	\begin{figure}
		\includegraphics[scale=0.03]{task2.jpg}
		\hfill
		\includegraphics[scale=0.03]{task3.jpg}
	\end{figure}
\end{frame}

\begin{frame}
	\frametitle{\subsecname}
	After every trial a \textit{vote} to the work done is given to the user.\\~\
	
	After the \textit{first} trail of every task the replay phase based on the waypoints and actions on gripper is shown to the users. IN this way the user understands where he can improve\\~\
	
	The starting position of the robot is always the same for all the users to unify the times.\\~\
	
	Every trial of the users were considered valid, even if a mistake were made by the user.
\end{frame}


\subsection{Post experiment phase}

\begin{frame}
	\frametitle{\subsecname}
	At the end of the experiment an evaluation questionnaire was filled by the users.\\~\
	
	It's divided in three parts:
	\begin{enumerate}
		\item personal informations as physical characteristics and confidence with pad
		\item some questions about mental and physical effort
		\item general questions about the experiment
	\end{enumerate}
\end{frame}

\subsection{Participants}

\begin{frame}
	\frametitle{\subsecname}
	The total number of participants is ten: seven male and three female. All of them are university graduates or students. Their ages are between $ 24 $	and $ 27 $ years old.\\~\
	
	Half of them never use a robot as \kuka{}.\\~\
	
	To obtain better and diversified results among the participants, the modality in which each user starts with the experiment is diversified. Half of them started with \te{}, the other with \kt{}.
\end{frame}

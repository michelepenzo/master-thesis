% Introduction section

\section{Experimental design}

\begin{frame}
	\frametitle{\secname}
	\begin{center}
		\includegraphics[scale=0.17]{users.png}\\~\
	\end{center} 
	% divisi in due completamente a caso
	% solo immagine della divisione --> randomizzato divisione degli utenti 
	To obtain better and diversified results among the participants, half of them started with \te{} teaching, while the others with \kt{} teaching.
\end{frame}

\begin{frame}
	\frametitle{\secname}
	% troppo lunga
	% dividere slide in due
	% 1) parlo dei task ---> immagine dei due task --> evidenzio che l'incastro è largo e stretto (img dall'alto --> righetta)
	% 2) parlo dei dati --> altra slide parlo dei dati raccolti --> utilizzo stack ros (img che ho messo nella tesi)
	% schema a blocchi --> confidenza --> esegue test --> questionario (segue le indicazioni di nasa tlx)
	Before the experiment, was asked to the users to take confidence with both the modalities. Every user performs this phase without seeing other users do the same. \\~\
	
	The experiment consists in two tasks in ascending order of difficulty repeated for three times. These tasks were repeated in \kt{} and \te{} and the users were rated with the data collected.\\~\
	
	\begin{center}
		\includegraphics[scale=0.17]{data.png}
	\end{center} 
\end{frame}

\begin{frame}
	\frametitle{\secname}
	% img task1  + video del replay
	% video dell'incastro --> replay tramite teleop
	In this figure the setup overview is shown: the \kuka{} and the shapes used for the task.\\~\

	 \begin{center}
		\includegraphics[scale=0.044]{setup.jpg}
	\end{center} 

\end{frame}
% experiment section

\section{Experimental design}

\begin{frame}
	\frametitle{Participants}
	% chi sono gli utenti, laureati, meta no robot, età
	% come sono stati divisi -> metà teleop metà kt
	% per ottenere dati diverisificati
	\begin{figure}
		\centering
		\includegraphics<1>[scale=0.17]{users_1.png}
		%\includegraphics<2>[scale=0.17]{users_2.png}
		\includegraphics<2>[scale=0.17]{users_3.png}
	\end{figure} 
\end{frame}

\begin{frame}
	\frametitle{Tasks}
	% 1) parlo dei task ---> immagine dei due task --> evidenzio che l'incastro è largo e stretto (img dall'alto --> righetta)
	
	\begin{figure}[H]
		\centering
		\includegraphics[scale=0.038]{task2.jpg}
		\hfill
		\includegraphics[scale=0.038]{task3.jpg}
	\end{figure}

\end{frame}

\begin{frame}
	\frametitle{Experimental flow}

	% fase di warm up con privacy polici, nessun utente vede gli altri utenti fare lo stesso
	% due task sin ordine crescente di difficoltà ripetuti per 3 volte in kt e tt
	% dopo ogni fase viene mostrata la fase di replay
	% viene fatto compilare un questionario alla fine nasa tlx, con anche dati su caratteristiche fisiche 
	\begin{figure}
		\centering
		\includegraphics<1>[scale=0.17]{experiment_1.png}
		\includegraphics<2>[scale=0.17]{experiment_2.png}
		\includegraphics<3>[scale=0.17]{experiment_3.png}
		\includegraphics<4>[scale=0.17]{experiment_4.png}
	\end{figure} 
\end{frame}
	
\begin{frame}
	\frametitle{Data collection}

	% parlo dati raccolti, cosa sono e a cosa servono
	% parlo dello stack ros open source
	\begin{figure}
		\centering
		\includegraphics[scale=0.18]{ros_sunrise.png}
	\end{figure} 


\end{frame}

%
%\begin{frame}
%	\frametitle{\secname}
%	% img task1  + video del replay
%	% video dell'incastro --> replay tramite teleop
%	In this figure the setup overview is shown: the \kuka{} and the shapes used for the task.\\~\
%
%	 \begin{center}
%		\includegraphics[scale=0.044]{setup.jpg}
%	\end{center} 
%
%\end{frame}


\begin{frame}
	\frametitle{Replay phase}	
	% video incastro tramite teleop di un utente
	\begin{center}
		\begin{flushright}
			Playback speed: 1x (normal)
		\end{flushright}
		\movie[width=10.6cm, height=6cm, repeat, poster, autostart]{}{./video/play.mp4}
	\end{center}
\end{frame}

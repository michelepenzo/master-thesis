% Introduction section

\section{Experimental design}

\begin{frame}
	\frametitle{\secname}
	Before the experiment, users were asked to take confidence with both the modalities. Every user perform this phase without seeing other users do the same. \\~\
	
	The experiment consists in two tasks in ascending order of difficulty repeated for three times. These tasks were repeated in \kt{} and \te{} and the users were rated with the data collected.\\~\
	
	\begin{center}
		\includegraphics[scale=0.17]{data.png}
	\end{center} 
\end{frame}

\begin{frame}
	\frametitle{\secname}
	In this figure the setup overview: the \kuka{} and on the left the shapes used for the tasks, and on the right the first task.\\~\

	 \begin{center}
		\includegraphics[scale=0.044]{setup.jpg}
	\end{center} 

\end{frame}


\begin{frame}
	\frametitle{\secname}

	\begin{center}
		\includegraphics[scale=0.17]{users.png}\\~\
	\end{center} 

	To obtain better and diversified results among the participants, half of them started with \te{} teaching, the other with \kt{} teaching.

\end{frame}

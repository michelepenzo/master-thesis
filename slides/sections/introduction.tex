% Introduction section

\section{Introduction}

% ------------------------------------------

\begin{frame}
\frametitle{\secname}
Within the industry 4.0, robots are increasingly exploited in production plants.\\~\

With the ambition to introduce robots into assembly lines the need to reconfigure the workspace requires faster modalities for robot reprogramming. \\~\
\end{frame}


\begin{frame}
\frametitle{\secname}
Actually, in automotive industry, welding and painting tasks are already highly automated.\\~\

Instead assembly tasks are mainly performed manually today and they are absolutely repetitive and they can be constantly changed.\\~\

These tasks are mainly:
\begin{itemize}
	\item pick and place
	\item peg into hole\\~\
\end{itemize}

\end{frame}

\begin{frame}
\frametitle{\secname}
To facilitate reprogramming of robots, the new paradigm which is used more frequently is \textit{PbD}: Programming by Demonstration.\\~\

\textit{Pbd} is often used with collaborative robots that are installed in industrial environments.\\~\

It's a technique for teaching a robot new behaviors by demonstrating the task through a sequence of commands.
\end{frame}


% ------------------------------------------
\subsection{Goals}

\begin{frame}
\frametitle{\subsecname}
	
From \textit{PbD} paradigm a comparison between two modalities was made to find the optimal method for teaching industrial assembly tasks.\\~\

The two modalities compared were:
\begin{itemize}[<+->]
	\item \textbf{\kt{} teaching}: the robot is gravity compensated and the user physically guides the robot within his workspace
	\item \textbf{\te{} teaching}: the user controls the robot with a \textbf{Ps4} pad
\end{itemize}
	
\end{frame}

\begin{frame}
\frametitle{\subsecname}

Before starting the work some research questions can be done:

\begin{itemize}[<+->]
	\item Which mode is preferred for ease of use?
	%\item Which one required more physical and mental effort?
	\item The two proposed approaches are said to be intuitive, but how much when they are used for assembly tasks in industry?
	\item There is a correlation between physical characteristics of the users and \kt{} teaching?
	\item users who have familiarity with the pad are better \te{} teaching?
\end{itemize}
\end{frame}





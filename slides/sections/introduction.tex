% Introduction section

\begin{frame}
	\frametitle{Summary}
	\begin{itemize}
		\item \textbf{Motivations}: within the Industry 4.0, collaborative robots with advanced features, more flexibility and safety are increasingly exploited in production plants. Cobot requires new reprogramming techniques, such as \kt{} and \te{}. \\~\
		
		% introduzione schema teleop specificatamente per assemblaggio
		% tramite user test voglio trovare risposta
		
		\item \textbf{Goals}: this work proposes a new \te{} scheme for teaching assembly tasks. Therefore, with an user experiment, the work wants to highlight the differences between the two modalities and to find the optimal method for teaching assembly tasks.
	\end{itemize}
\end{frame}

\begin{frame}
	\frametitle{Summary}
	We formulate some preliminary questions:
	\begin{itemize}
		\item Which mode is preferred for easy of use?
		\item The two proposed approaches are said to be intuitive, but how much when they are applicable for industrial assembly task programming?
		\item Are the physical characteristics of the users impacting on \kt{} teaching?
		\item Are users who are familiar with the pad better in \te{} than the other users?
	\end{itemize}
\end{frame}


\begin{frame}{Table of contents}
	\setbeamertemplate{section in toc}[sections numbered]
	\tableofcontents[hideallsubsections]
\end{frame}

\section{Introduction}

% ------------------------------------------

\begin{frame}
	\frametitle{\secname}
	In automotive industry welding and painting tasks are already highly automated, while assembly tasks in industry are mainly performed manually today.\\~\
	
	For those tasks that are performed by robot, the default modality for teaching assembly tasks is \kt{} teaching.\\~\
	
	The use of a collaborative robot and \kt{} teaching doesn't allow us to fully use the collaborative features.
	% img dove robot sono già utilizzati
	% ur che carica/scarica torni
	% uno di assemblaggio
\end{frame}

\begin{frame}
	\frametitle{\secname}
	% proponiamo .. setup con kuka iiwa, collaborativo, best robot, disponibile in ICE
	
	For these reasons and inspired by other works, a new \te{} modality for teaching assembly tasks was implemented using the \kuka{} and a \textbf{PS4 pad}.\\~\\~\\~\


	\begin{figure}
		\centering
		\includegraphics[scale=0.12]{ps4_pad_lines.png}
	\end{figure}


\end{frame}

\begin{frame}
	\frametitle{\secname}	
	
	\begin{center}
		\begin{flushright}
			Playback speed: 1x (normal)
		\end{flushright}
		\movie[width=10.6cm, height=6cm, repeat, poster, autostart]{}{./video/pic_in_pic.mp4}
	\end{center}
\end{frame}

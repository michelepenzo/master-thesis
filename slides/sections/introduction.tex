% Introduction section

\begin{frame}
	\frametitle{Summary}
	\begin{itemize}
		\item \textbf{Motivations}: within the Industry 4.0, collaborative robot with advanced features, more flexibility and safety are increasingly exploited in production plants. Cobot requires new reprogramming techniques, such as \kt{} and \te{}. \\~\
		
		\item \textbf{Goals}: this work proposes a new \te{} modality for teaching assembly tasks. Therefore, with an user experiment, the work wants to highlight the differences between the two modalities for teaching assembly tasks.
	\end{itemize}
\end{frame}

\begin{frame}
	\frametitle{Summary}
	We formulate some preliminary questions:
	\begin{itemize}
		\item Which mode is preferred for easy of use?
		\item The two proposed approaches are said to be intuitive, but how much when they are applicable for industrial assembly task programming?
		\item Are the physical characteristics of the users impacting on \kt{} teaching?
		\item Are users who are familiar with joypad better at \te{} than the other users?
	\end{itemize}
\end{frame}


\begin{frame}{Table of contents}
	\setbeamertemplate{section in toc}[sections numbered]
	\tableofcontents[hideallsubsections]
\end{frame}

\section{Introduction}

% ------------------------------------------

\begin{frame}
	\frametitle{\secname}
	In automotive industry welding and painting tasks are already highly automated, while assembly tasks in industry are mainly performed manually today.\\~\
	
	For those tasks that are performed by robots, the default modality for teaching assembly tasks is \kt{} teaching.\\~\
	
	The use of a collaborative robot such as \kuka{} and \kt{} teaching doesn't allow us to fully use the collaborative features.
\end{frame}

\begin{frame}
	\frametitle{\secname}	
	For these reasons and inspired by other works, a new modality for teaching assembly tasks was implemented.\\~\
	
	\Te{} teaching provides a way to control robot using a \textbf{PS4 pad} that allows us to have more control options.\\~\

	It allows to have a safe remote control of the robot through the use of buttons and analogs. It also have a force feedback features based on the vibration of the pad.

\end{frame}

\begin{frame}
	\frametitle{\secname}	
	This video shows how \te{} teaching is used to perform the task.\\~\
	
	\begin{center}
		\movie[width=8cm, height=4.5cm, repeat, poster, autostart]{}{./video/pic_in_pic.mp4}
	\end{center}
\end{frame}

% conclusi section

\section{Conclusions}

% ------------------------------------------

\begin{frame}
\frametitle{\secname}
	The modalities analyzed in this work have substantial differences. The first difference to highlight is the speed and naturalness with how the tasks can be taught in \kt{} opposed to \te{}.\\~\
	
	We can say that \kt{} teaching can be used when you want to teach simple tasks, such as pick and place.\\~\

	But it's necessary to take into account the physical characteristics of the users.	
\end{frame}


\begin{frame}
	\frametitle{\secname}
	To overcome this problem, \te{} is therefore used because unifies the differences between the physical characteristics.\\~\

	It offers the possibility of exploiting more functionalities of the collaborative robot.\\~\
	
	But it introduces greater slowness but precision. We can therefore say that this can be used for difficult tasks that require precision.

\end{frame}

\begin{frame}
	\begin{center}
		{\Large Thank you for your interest and attention.}
	\end{center}
\end{frame}

% conclusi section

\section{Conclusions}

% ------------------------------------------

\begin{frame}
	\frametitle{\secname}
	As mentioned, many aspects have to be considered and it isn't easy to find the best modality.\\~\

	\Kt{} introduces inequality among users due to physical characteristics.\\~\
	
	Furthermore, thanks to its simplicity \kt{} teaching can be used an user wants to teach simple tasks.
\end{frame}


\begin{frame}
	\frametitle{\secname}
	To overcome the problem of different physical characteristics, \te{} is used because unifies the differences.\\~\
	
	Furthermore, the abilities of the users can be reused and trained for better performances.\\~\
	
	We can for this reason say that \te{} teaching can be used for difficult tasks that require precision as it offers the possibility of exploiting more functionalities of the collaborative robot.

\end{frame}

\begin{frame}
	\begin{center}
		{\Large Thank you for your attention.}
	\end{center}
\end{frame}

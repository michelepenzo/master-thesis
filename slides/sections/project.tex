% The project section

\section{The project}

% ------------------------------------------
\subsection{Setup overview}

\begin{frame}
\frametitle{\subsecname}

The \kuka{} is a robot with collaborative features. It has a $ 14kg $ payload and an action range from $ 800mm $ to $ 820mm $. \\~\
\begin{columns}
	\begin{column}{0.45\textwidth}
		Moreover is has a media flange, with and internal wiring that is helpful to attach a lot of different tools.\\~\
		
		In our case the tool select is an electric parallel gripper with 2 self-centering two jaw with a total gripping force of $ 210N $.
	\end{column}
	\begin{column}{0.45\textwidth}
		\centering
		\includegraphics[scale=0.8]{kuka_lbr_iiwa.png}
	\end{column}
\end{columns}
\end{frame}


\begin{frame}
\frametitle{\subsecname}
Usually the \kuka{} is programmed using the KUKA's Sunrise Workbench platform and its Java API's.\\~\

Instead in this work an open source stack compatible with ROS has been uses.\\~\

It provides a Java Robotic Application that establishes a point to point connection via ROS to the machine connected via Ethernet to the robot cabinet.\\~\

\begin{center}
	\includegraphics[scale=0.4]{ros_sunrise.png}
\end{center}

\end{frame}

\begin{frame}
\frametitle{\subsecname}
The machine with ROS installed, will be able to send and receive ROS messages to and from the Robotic Application.\\~\

The robot can be controlled using Python, C++ or via commands sent from console.\\~\

Using ROS there are some functionalities already implemented as \texttt{service}, \texttt{topic} and \texttt{action}.

\end{frame}

% ------------------------------------------
\subsection{Project Implementation}

\begin{frame}
\frametitle{\subsecname}

One goal of the thesis is to evaluate different modalities to teach quickly and simply assembly taks in industry.\\~\

For each of the two modalities a way to save waypoints and actions on the gripper has been implemented.\\~\

The actions were captured and saved in a \texttt{.csv} file. The sequence of waypoints and actions can be replicated by the robot.


\end{frame}


% ------------------------------------------
\subsection{Teach by demonstration}

\begin{frame}
\frametitle{\subsecname}

As described in introduction, teach by demonstration or also called \kt{} is a way to move the robot in gravity compensation.\\~\

The gravity compensation mode has been implemented using \textit{joint impedance}, then for every joint a stiffness and damping value has been set.\\~\

After changing the control mode in joint impedance, the robot seems falls and a small force contrary must be carried out to keep it up.

\end{frame}


\begin{frame}
\frametitle{\subsecname}
On media flange there is a button that was dedicated for saving commands.\\~\

When the button is pressed according to the duration of the pressure, an action is recorded:
\begin{itemize}
	\item one click for save a waypoint
	\item $ 2 $ seconds pressure for save the action on the gripper
	\item $ 5 $ seconds pressure to exit from teaching program
\end{itemize}

\end{frame}

% ------------------------------------------
\subsection{Teach with remote control}

\begin{frame}
\frametitle{\subsecname}

Remote control, or \te{} indicates the movement of the robot at a distance. In our case, the \kuka{} was controlled with a Ps4pad.\\~\

In \textit{full task space} mode the robot will move respect to the base frame using linear and angular velocity on $ x $, $ y $ and $ z $:
\begin{itemize}
	\item with \textbf{linear} velocity the robot will move the pose of the robot
	\item with \textbf{angular} velocity the robot will move the orientation of the EE\\~\
\end{itemize}

For these movements, the analogs of the pad were used. Instead, for security, the analogs must be used with the continued use of two deadman buttons.
\end{frame}


\begin{frame}
\frametitle{\subsecname}
The buttons used to do actions on robot were:

\begin{itemize}
	\item $ \times $: close or open gripper, therefore save action and the actual pose
	\item $ \triangle $: save actual pose
	\item $ \square $: change to control mode from position to cartesian impedance, and vice versa\\~\
\end{itemize}

The possibility to change the control mode to cartesian impedance allows the users to be more relaxed for precise movements.\\~\

Therefore, when an external force grater than a preset value is detected the \textbf{pad} start vibrating.
\end{frame}


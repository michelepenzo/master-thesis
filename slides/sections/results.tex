% Results discussion section

\section{Results discussion}

% ------------------------------------------

\begin{frame}
\frametitle{\secname}
From the questionnaire many of the data collected were analyzed.\\~\

Especially from the questions in the first part of the questionnaire it's possible to spot the differences between the two modalities.\\~\

Therefore users had to answer questions related to physical and mental effort in both the modalities perceived during the experiment.
	
\end{frame}


\begin{frame}
\frametitle{\secname}
The efforts in the two modalities were analyzed separately to understand which data could be extracted.\\~\

From these data was possible to create different groups based on personal characteristics.\\~\

The two main classifications that have been determined based on:
\begin{itemize}
	\item physical characteristics
	\item confidence with the pad
\end{itemize}
\end{frame}

\begin{frame}
\frametitle{\secname}
How was the group based on \textbf{personal characteristics} created?\\~\

In questionnaire was asked the weight and height keeping a range between $ 10cm $ and $ 10kg $. Two macro groups, both of them composed by five users, were composed:
\begin{itemize}
	\item \textbf{LBU group}: composed by users which height is less or equal than $ 70kg $ and which height it’s less or equal than $ 180cm $
	\item \textbf{TBU group}: composed by users with weight grater that $ 70kg $ and height grater than $ 170cm $.
\end{itemize}

\end{frame}


\begin{frame}
	\frametitle{\secname}
	How was the group based on \textbf{confidence with the pad} created?
	\\~\
	
	In questionnaire was asked with how frequency the pad is used from the users. These users are called:
	\begin{itemize}
		\item \textbf{RP}: who replied more than once a week and at least once a month.
		\item \textbf{CP}: who replied that they use the pad once a year or they never used it. Also all users who have used the pad in the past are in this classification.
	\end{itemize}
\end{frame}
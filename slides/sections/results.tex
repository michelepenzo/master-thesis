% Results discussion section

\section{Results discussion}

% ------------------------------------------

\begin{frame}
	\frametitle{\secname}	
	For the analysis the users were divided into groups and their characteristics were analyzed.\\~\
	\begin{center}
		\includegraphics[scale=0.17]{users_by_group.png}
	\end{center}

	For reasons of time, only two of the research questions are demonstrated through data. For a more complete and in-depth explanation, refer to the thesis.
\end{frame}

\begin{frame}
	\frametitle{\secname}
	\textbf{There is a correlation between physical characteristics of the users and \kt{} teaching?} \\~\

	\begin{itemize}
		\item An analysis on the forces applied on the EE by the users of the two groups was made.\\~\
		\item Therefore, the questionnaire asked the \textit{physical effort} required by users to perform the tasks.
	\end{itemize}
	
\end{frame}

\begin{frame}
	\frametitle{\secname}
	This figure shows the difference of the forces applied by the groups while they were performing the tasks in \kt{} teaching. The force is in $ N $.
	\begin{center}
		\includegraphics[scale=0.15]{wrench_all_kt.png}
	\end{center}
\end{frame}

\begin{frame}
	\frametitle{\secname}
	Therefore, the previous concept is supported by the answers in the questionnaire.\\~\
	
	The averages of the physical effort of the users are: $ 5,4 $ for \smallusers{} group, and $ 3,8 $ for \bigusers{} group.\\~\
	
	Both personal opinions and scientific data show us that there is a correlation between \kt{} teaching and physical characteristics.
\end{frame}

\begin{frame}
	\frametitle{\secname}
	\textbf{Users who have familiarity with the pad are better with \te{} teaching?}\\~\
	
	\begin{itemize}
		\item Time and distance to complete the tasks for both groups were analyzed.\\~\
		\item Therefore, the answers to the question regarding the \textit{mental effort} required by users to perform the tasks were analyzed.
	\end{itemize}
\end{frame}

\begin{frame}
	\frametitle{\secname}
	Only the times (in $ seconds $) about task t1 are shown, for task t2 the values are	similar.
	
	\begin{center}
		\includegraphics[scale=0.15]{time_groups_t2.png}
	\end{center}
%\note{test}
\end{frame}

\begin{frame}
	\frametitle{\secname}
	The concept of time and distance to complete the task is also related with the concept of ratio and precision of waypoints.\\~\
	
	Furthermore, from the previous figure it's possible to notice how \te{} teaching is always slower than \kt{} teaching.\\~\
	
	Instead the mental efforts in \te{} teaching are: $ 5,6 $ for \smallusers{} and $ 6 $ for \bigusers{}.\\~\
\end{frame}
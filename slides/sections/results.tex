% Results discussion section

\section{Results discussion}

% ------------------------------------------

\begin{frame}
	\frametitle{\secname}
	From the questionnaire many of the data collected were analyzed.\\~\
	
	Especially from the questions in the first part of the questionnaire it's possible to spot the differences between the two modalities.\\~\
	
	Therefore users had to answer questions related to physical and mental effort in both the modalities perceived during the experiment.
\end{frame}


\begin{frame}
	\frametitle{\secname}
	The efforts in the two modalities were analyzed separately to understand which data could be extracted.\\~\
	
	From these data was possible to create different groups based on personal characteristics.\\~\
	
	The two main classifications that have been determined based on:
	\begin{itemize}
		\item physical characteristics
		\item confidence with the pad
	\end{itemize}
\end{frame}

\begin{frame}
	\frametitle{\secname}
	How was the group based on \textbf{personal characteristics} created?\\~\
	
	In questionnaire was asked the weight and height keeping a range between $ 10cm $ and $ 10kg $. Two macro groups, both of them composed by five users, were composed:
	\begin{itemize}
		\item \textbf{LBU group}: composed by users which height is less or equal than $ 70kg $ and which height it’s less or equal than $ 180cm $
		\item \textbf{TBU group}: composed by users with weight grater that $ 70kg $ and height grater than $ 170cm $.
	\end{itemize}
\end{frame}


\begin{frame}
	\frametitle{\secname}
	How was the group based on \textbf{confidence with the pad} created?
	\\~\
	
	In questionnaire was asked with how frequency the pad is used from the users. These users are called:
	\begin{itemize}
		\item \textbf{RP}: who replied more than once a week and at least once a month.
		\item \textbf{CP}: who replied that they use the pad once a year or they never used it. Also all users who have used the pad in the past are in this classification.
	\end{itemize}
\end{frame}


\begin{frame}
	\frametitle{\secname}
	The work was mainly concentrated on the division into groups based on physical characteristics.\\~\

	Therefore for each group the differences between the modalities were highlighted.\\~\
	
	It's necessary to make an initial clarification. The \smallusers{} group is composed by five users which are defined as \regular{}. This was noticed only after the analysis that was made on users by physical characteristics.
	
\end{frame}

\begin{frame}
	\frametitle{\secname}
	The first difference that was noticed was relative to time and distance for complete tasks in both the modalities. Instead, there is a correlation between time and distance. \\~\
	
	Starting with these data, is possible to affirm that the phase relative to \te{} teaching is always slower than the phase relative to \kt{} teaching. \\~\
	
	This aspect is independent from the division into groups.
\end{frame}

\begin{frame}
	\frametitle{\secname}
	Instead, from the next figures it's possible to see how the users of \smallusers{} group are faster than \bigusers{}. This is due to the fact that four users in \smallusers{} group are \regular.\\~\
	
	Instead, the users of \bigusers{} group are actually faster than the users of the other group. This is due to the physical characteristics of each user.
\end{frame}

\begin{frame}
	\frametitle{\secname}
	The time is calculated in seconds and the distance in meters. Only the times and distances about task $ t1 $ are shown, for task $ t2 $ the values are similar.
	
	\begin{figure}[ht]
		\centering
		\begin{minipage}[b]{0.495\linewidth}
			\centering
			\includegraphics[width=\textwidth]{time_groups_t2.png}
		\end{minipage}
		\begin{minipage}[b]{0.495\linewidth}
			\centering
			\includegraphics[width=\textwidth]{distance_groups_t2.png}
		\end{minipage}
		\caption{Time and distance for completing task $ t1 $}
	\end{figure}
\end{frame}

\begin{frame}
	\frametitle{\secname}
	Related to the concept of time and distance to perform the task, there are the data relative to the \texttt{cartesian\_wrench}. These collect data shows the force applied on the EE during the experiment.\\~\
	                                                                                                                                                                          
	It's possible to see how obviously the forces applied in the two modalities are completely different: in \te{} the forces are practically nil, instead in \kt{} there are more interaction forces.
	
\end{frame}

\begin{frame}
	\frametitle{\secname}
	Force is expressed in $ N $ and only the force calculated on $ z $ is shown. The same graphs are similar for $ x $ and $ y $.
	\begin{figure}[ht]
		\begin{minipage}[b]{0.495\linewidth}
			\centering
			\includegraphics[width=\textwidth]{wrench_tt.png}
		\end{minipage}
		\begin{minipage}[b]{0.495\linewidth}
			\centering
			\includegraphics[width=\textwidth]{wrench_kt.png}
		\end{minipage}
		\caption{Difference between force on $ z $ in \te{} e \kt}
	\end{figure}
\end{frame}

\begin{frame}
	\frametitle{\secname}

	An analysis was made on the force exerted on the robot by the users divided in groups during the experiment. \\~\
	
	Given the difference in time and distance between the two groups, the strength to perform the tasks was also analyzed.\\~\
	
	Has been noticed that the users of the \smallusers{} group user more force to perform the task. This also tells us why these users find the kt phase more tiring than the other group of users.
	
\end{frame}

\begin{frame}
	\frametitle{\secname}
	These figure shown the difference between the two groups performing the tasks in \kt{} teaching. It's possible to notice how the force (in $ N $) on $ z $ and $ x $ are higher for the \smallusers{} group. On $ y $ there aren't variations as no large movements are made.\\~\
	
	\begin{figure}[ht]
		\centering
		\includegraphics[scale=0.15]{wrench_all_kt.png}
		\caption{Force on $ x,y,z $ during \kt{} teaching}
	\end{figure}
	
\end{frame}


\begin{frame}
	\frametitle{\secname}
	Instead, related to the concept of \regular{} that was shown that they perform the task faster there is the concept of ratio of waypoints.\\~\
	
	In fact, since the robot as to replay the sequence thought, some waypoints must be taken. The minimum number is 16, the optimal number is 18.\\~\
	
	It's also noted how in \te{} the number of waypoints are minor respect to \kt{}: there isn't difference between the two groups.
\end{frame}

\begin{frame}
	\frametitle{\secname}
	The users classified as \regular{} during \te{} take waypoints more precise and focused on the shape. Also, these users take fewer waypoints. \\~\
	
	This can be translated as a faster and more effective replay phase.\\~\

	Instead, in \kt{} there isn't difference between the two groups where a lot of waypoints were taken out of the center of the figure.
\end{frame}

\begin{frame}
	\frametitle{\secname}
	
	
\end{frame}


\begin{frame}
	\frametitle{\secname}
	
	
\end{frame}

\begin{frame}
	\frametitle{\secname}
	
	
\end{frame}

\begin{frame}
	\frametitle{\secname}
	
	
\end{frame}


\begin{frame}
	\frametitle{\secname}
	
	
\end{frame}

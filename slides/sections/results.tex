% Results discussion section

\section{Results discussion}

% ------------------------------------------

\begin{frame}
	\frametitle{\secname}
	From the questionnaire many of the data collected were analyzed.\\~\
	
	Especially, from the questions in the first part of the questionnaire, it's possible to spot the differences between the two modalities.\\~\
	
	In fact the users had to answer questions related to physical and mental effort perceived during the experiment in both the modalities.
\end{frame}


\begin{frame}
	\frametitle{\secname}
	The efforts in the two modalities were analyzed separately to understand which data could be extracted.\\~\
	
	From these data was possible to create different groups based on personal characteristics.\\~\
	
	The two main classifications that have been determined based on:
	\begin{itemize}
		\item physical characteristics,
		\item confidence with the pad.
	\end{itemize}
\end{frame}

\begin{frame}
	\frametitle{\secname}
	How was the group based on \textbf{personal characteristics} created?\\~\
	
	In questionnaire was asked the weight and height keeping a range between $ 10kg $ and $ 10cm $. Two macro groups, both of them composed by five users, were composed:
	\begin{itemize}
		\item \textbf{LBU group}: composed by users which height is less or equal than $ 70kg $ and which height it’s less or equal than $ 180cm $,
		\item \textbf{TBU group}: composed by users with weight greater that $ 70kg $ and height grater than $ 170cm $.
	\end{itemize}
\end{frame}


\begin{frame}
	\frametitle{\secname}
	How users were classified based on their \textbf{confidence with the pad}?
	\\~\
	
	In questionnaire was asked with how frequency the pad is used from the users. These users are called:
	\begin{itemize}
		\item \textbf{RP}: who replied more than once a week and at least once a month,
		\item \textbf{CP}: who replied that they use the pad once a year or they never used it. Also all users who have used the pad in the past are in this classification.
	\end{itemize}
\end{frame}


\begin{frame}
	\frametitle{\secname}
	The work was mainly concentrated on the division into groups based on physical characteristics.\\~\

	Therefore for each group the differences between the modalities were highlighted.\\~\
	
	It's necessary to make an initial clarification. The \smallusers{} group is composed by four users which are defined as \regular{}. This was noticed only after the analysis that was made on users by physical characteristics.
	
\end{frame}

\begin{frame}
	\frametitle{\secname}
	The first difference that was noticed was relative to time and distance for complete tasks in both the modalities. Instead, there is a correlation between time and distance. \\~\
	
	Starting with these data, is possible to affirm that the phase relative to \te{} teaching \textit{is always slower} than the phase relative to \kt{} teaching. \\~\
	
	This aspect is \textit{independent} from the division into groups.
\end{frame}

\begin{frame}
	\frametitle{\secname}
	Instead, from the next figures it's possible to see how the users of \smallusers{} group are faster in \te{} teaching than \bigusers{} users. This is due to the fact that four users in \smallusers{} group are \regular.\\~\
	
	Instead, the users of \bigusers{} group are actually faster in \kt{} teaching than the users of the other group. This is due to the physical characteristics of each user.
\end{frame}

\begin{frame}
	\frametitle{\secname}
	The time is calculated in seconds and the distance in meters. Only the times and distances about task $ t1 $ are shown, for task $ t2 $ the values are similar.
	
	\begin{figure}[ht]
		\centering
		\begin{minipage}[b]{0.495\linewidth}
			\centering
			\includegraphics[width=\textwidth]{time_groups_t2.png}
		\end{minipage}
		\begin{minipage}[b]{0.495\linewidth}
			\centering
			\includegraphics[width=\textwidth]{distance_groups_t2.png}
		\end{minipage}
		\caption{Time and distance for completing task $ t1 $}
	\end{figure}
\end{frame}

\begin{frame}
	\frametitle{\secname}
	Related to the concept of time and distance to perform the task, there are the data relative to the \texttt{cartesian\_wrench}. These collect data shows the force applied on the EE during the experiment.\\~\
	                                                                                                                                                                          
	It's possible to see how obviously the forces applied in the two modalities are completely different: in \te{} the forces are practically nil, instead in \kt{} there are more interaction forces.
	
\end{frame}

\begin{frame}
	\frametitle{\secname}
	Only the force calculated in $ N $ on $ z $ is shown. The same graphs are similar for $ x $ and $ y $.
	\begin{figure}[ht]
		\begin{minipage}[b]{0.495\linewidth}
			\centering
			\includegraphics[width=\textwidth]{wrench_tt.png}
		\end{minipage}
		\begin{minipage}[b]{0.495\linewidth}
			\centering
			\includegraphics[width=\textwidth]{wrench_kt.png}
		\end{minipage}
		\caption{Difference between force on $ z $ in \te{} e \kt{} teaching}
	\end{figure}
\end{frame}

\begin{frame}
	\frametitle{\secname}
	Given the difference in time and distance between the two groups, the force to perform the tasks was also analyzed.\\~\

	An analysis was made on the force exerted on the EE by the users divided in groups during the experiment. \\~\
		
	Has been noticed that the users of the \smallusers{} group use \textit{more force} to perform the task. This also tells us why these users find the \kt{} phase more tiring than the other group of users.
	
\end{frame}

\begin{frame}
	\frametitle{\secname}
	These figure shown the difference of force applied by the groups while they were performing the tasks in \kt{} teaching. It's possible to notice how the force (in $ N $) on $ x,z $ are higher for the \smallusers{} group. On $ y $ there aren't variations as no large movements are made.\\~\
	
	\begin{figure}[ht]
		\centering
		\includegraphics[scale=0.15]{wrench_all_kt.png}
		\caption{Force on $ x,y,z $ during \kt{} teaching}
	\end{figure}
	
\end{frame}


\begin{frame}
	\frametitle{\secname}
	Instead, related to the concept of \regular{} these users perform the task faster. It's also introduced the concept of ratio of waypoints.\\~\
	
	In fact, since the robot has to replay the sequence thought, some waypoints must be taken. The minimum number of waypoints is $ 16 $, the optimal number is $ 18 $.\\~\
	
	It's also noted how in \te{} the number of waypoints are minor respect to \kt{}.
\end{frame}

\begin{frame}
	\frametitle{\secname}
	The users classified as \regular{} during \te{} take waypoints more precise and focused on the shapes. Also, these users take fewer waypoints. \\~\
	
	This can be translated as a faster and more effective replay phase.\\~\

	Instead, in \kt{} there aren't differences between the two groups. Respect to \te{}, in \kt{} a lot of waypoints were taken out of the center of the figure.
\end{frame}

\begin{frame}
	\frametitle{\secname}
	Also the data relative to physical and mental effort were analyzed to focus on how much personal characteristics affect \te{} and \kt{} teaching.\\~\
	
	From the data about \textbf{mental effort in \te{}}, there isn't a lot of difference between the two groups. The \smallusers{} has a lower value, that is due to the fact that these users have confidence with the pad.\\~\
	
	We can say that in general \te{} isn't very mentally tiring or stressful, especially because there isn't interaction with the robot. 
\end{frame}


\begin{frame}
	\frametitle{\secname}
	From the data about \textbf{mental effort in \kt{}}, there is a little difference between the two groups. This is mainly due to the fact that more than half of the \smallusers{} group never use a root before this experiment.\\~\
	
	So, \kt{} teaching is rated as easy by the users who are familiar with the robots.\\~\
	
	Instead, both the values of the mental effort given by the two groups are lower than the values of mental effort in \te.
	
\end{frame}

\begin{frame}
	\frametitle{\secname}
	The two values obtained from the questionnaire are identical for \textbf{physical effort in \te{}} because there isn't interaction with the robot.\\~\
	
	Regarding \textbf{physical effort in \kt{}}, there is a substantial difference between the two groups. This allows us to highlight how the characteristic differences affect \kt{} in terms of strength.\\~\ 
	
	%The concept of strength, as mentioned before, is also related to the time required to perform the task.\\~\
	
	Instead, the values of physical effort in \kt{} are always slower than the values in \te.
\end{frame}

%\begin{frame}
%	\frametitle{\secname}
%	% pros and cons
%\end{frame}
%
%\begin{frame}
%	\frametitle{\secname}
%	% pros and cons
%\end{frame}


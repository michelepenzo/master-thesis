% Results discussion section

\section{Results discussion}

% ------------------------------------------

\begin{frame}
	\frametitle{\secname}	
	For the analysis the users were divided into groups using \textit{BMI}.\\~\

	% dico che sono divisi in base al BMI, quindi anche la soglia --> questionario peso e altezza
	% all'interno dei gruppi segnalo che ci sono n rp
	\begin{center}
		\includegraphics[scale=0.17]{users_by_group.png}\\~\
	
		\begin{table}[]
			\begin{tabular}{c||l}
				\textbf{Abbreviation} & \textbf{Extended form}\\
				\hline \hline
				\textbf{LBU} & \textbf{L}ow \textbf{B}MI \textbf{U}sers\\
				\textbf{HBU} & \textbf{H}igh \textbf{B}MI \textbf{U}sers\\
				\hline
				\textbf{RP}  & \textbf{R}egular \textbf{P}layers\\
				\textbf{CP}  & \textbf{C}asual \textbf{P}layers
			\end{tabular}
		\end{table}
	
	\end{center}

\end{frame}

\begin{frame}
	\frametitle{\secname}
	For reasons of time, in this presentation only two questions are shown as they are interesting in terms of results achieved.\\~\

	\begin{itemize}
		\color{gray}
		\item Which mode is preferred for easy of use?
		\item The two proposed approaches are said to be intuitive, but how much when they are applicable for industrial assembly task programming?
		\color{black}
		\item Are the physical characteristics of the users impacting on \kt{} teaching?
		\item Are users who are familiar with the pad better in \te{} than the other users?
	\end{itemize}
\end{frame}

\begin{frame}
	\frametitle{\secname}
	\textbf{Are the physical characteristics of the users impacting on \kt{} teaching?} \\~\

	\begin{itemize}
		\item An analysis on the forces applied on the EE by the users of the two groups was made.\\~\
		\item Therefore, the questionnaire asked the \textit{physical effort} required by users to perform the tasks.
	\end{itemize}
	
\end{frame}

\begin{frame}
	\frametitle{\secname}
	This figure shows the difference of the forces applied by the groups while they were performing the tasks in \kt{} teaching. The force is in $ N $.
	\begin{figure}
		\centering
		\begin{tikzpicture}
			\node(a){\includegraphics[scale=0.15]{wrench_all_kt.png}};
			\draw[red, thick, ->] (4,-1) -- (5.2,-1);
			\draw[blue, thick, ->] (4,-1) -- (4,0.2);
			\draw[green, thick, ->] (4,-1) -- (5,0);
			\node[] at (5.5, -1) {x};
			\node[] at (4, 0.5) {z};			
			\node[] at (5.25, 0.25) {y};
		\end{tikzpicture}   
	\end{figure}
\end{frame}

\begin{frame}
	\frametitle{\secname}
	Therefore, the previous concept is supported by the answers in the questionnaire.\\~\
	
	The averages of the physical effort of the users are: $ 5,4 $ for \smallusers{} group and $ 3,8 $ for \bigusers{} group.\\~\
	
	Both personal opinions and scientific data show us that there is a correlation between \kt{} teaching and physical characteristics.
\end{frame}

\begin{frame}
	\frametitle{\secname}
	\textbf{Are users who are familiar with the pad better in \te{} than the other users?}\\~\
	
	\begin{itemize}
		\item Time and distance to complete the tasks for both groups were analyzed.\\~\
		\item Therefore, the answers to the question regarding the \textit{mental effort} required by the users to perform the tasks were analyzed.
	\end{itemize}
\end{frame}

\begin{frame}
	\frametitle{\secname}
	Only the times (in $ seconds $) about task $ t1 $ are shown, for task $ t2 $ the values are	similar.
	
	\begin{figure}
		\centering
		\includegraphics[scale=0.15]{time_groups_t2.png}
	\end{figure}
\end{frame}

\begin{frame}
	\frametitle{\secname}
	% distanza -> tempo -> dimestichezza
	% waypoints e ratio
	
	\begin{figure}
		\centering
		\begin{tikzpicture}
			\node(a) {\includegraphics[scale=0.13]{waypoints_tt_t1.png}};
			\node[] at (4, 0) {LBU};
			\node[] at (4, -0.5) {HBU};
			\node at (4.5, 0) [circle, red, fill,inner sep=1.5pt]{};
			\node at (4.5, -0.5) [circle, blue, fill,inner sep=1.5pt]{};
		\end{tikzpicture}   
	\end{figure}
	
	%The concept of time and distance to complete the task is also related with the concept of ratio and precision of waypoints.\\~\
	
	The mental efforts in \te{} teaching are: $ 5,6 $ for \smallusers{} and $ 6 $ for \bigusers{}. Also, the data show us how users confident with the pad are better in \te{}.\\~\
	
	Furthermore, from the previous figure it's possible to notice how \te{} teaching is always slower than \kt{} teaching.
\end{frame}
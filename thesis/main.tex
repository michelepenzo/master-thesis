\documentclass[
11pt, % The default document font size, options: 10pt, 11pt, 12pt
oneside, % Two side (alternating margins) for binding by default, uncomment to switch to one side
english, % ngerman for German
singlespacing, % Single line spacing, alternatives: onehalfspacing or doublespacing
%draft, % Uncomment to enable draft mode (no pictures, no links, overfull hboxes indicated)
%nolistspacing, % If the document is onehalfspacing or doublespacing, uncomment this to set spacing in lists to single
%liststotoc, % Uncomment to add the list of figures/tables/etc to the table of contents
%toctotoc, % Uncomment to add the main table of contents to the table of contents
%parskip, % Uncomment to add space between paragraphs
headsepline, % Uncomment to get a line under the header
%chapterinoneline, % Uncomment to place the chapter title next to the number on one line
%consistentlayout, % Uncomment to change the layout of the declaration, abstract and acknowledgements pages to match the default layout
]{MastersDoctoralThesis} % The class file specifying the document structure

\usepackage[utf8]{inputenc} % Required for inputting international characters
\usepackage[T1]{fontenc} % Output font encoding for international characters
\usepackage{mathpazo} % Use the Palatino font by default
\usepackage[backend=bibtex, natbib=true]{biblatex}
\usepackage[autostyle=true]{csquotes}
\usepackage{float}
\usepackage{graphics}
\usepackage{tikz}
\usepackage{scalerel,amssymb}
\usepackage{subcaption}
\usepackage{hyperref}
\usepackage{dirtree}

\addbibresource{references.bib}
\graphicspath{{figures/}} 

% some commands
\newcommand*\circled[1]{\tikz[baseline=(char.base)]{ \node[shape=circle,draw,inner sep=1pt] (char) {#1};}}
\def\checkmark{\tikz\fill[scale=0.4](0,.35) -- (.25,0) -- (1,.7) -- (.25,.15) -- cycle;}
\newcommand{\kuka}{\textrm{KUKA LBR IIWA}}

\newcommand{\kt}{kinesthetic}
\newcommand{\te}{teleoperation}
\newcommand{\Kt}{Kinesthetic}
\newcommand{\Te}{Teleoperation}

\newcommand{\rep}{$ 3 $}
\newcommand{\nobj}{$ 3 $}

\newcommand{\smallusers}{\textbf{LBU}}
\newcommand{\bigusers}{\textbf{TBU}}
\newcommand{\regular}{\textbf{RP}}
\newcommand{\casual}{\textbf{RP}}

%	MARGIN SETTINGS

\geometry{
	paper=a4paper, % Change to letterpaper for US letter
	inner=3cm, % Inner margin
	outer=3cm, % Outer margin
	%bindingoffset=.5cm, % Binding offset
	top=2cm, % Top margin
	bottom=2cm, % Bottom margin
	%showframe, % Uncomment to show how the type block is set on the page
}

\setlength\parindent{0pt}

%	THESIS INFORMATION

\thesistitle{Towards an optimal method for teaching industrial assembly tasks using collaborative robots: teleoperation vs kinesthetic}
\supervisor{Dr. Diego \textsc{Dall'Alba}}

\degree{Master of Science in Computer Science}
\author{Michele \textsc{Penzo}\\ VR439232}
\keywords{}
\university{\href{https://www.univr.it/en/home}{University of Verona}}
\department{\href{https://www.di.univr.it/?lang=en}{Department of Computer Science}}
\group{\href{https://www.icelab.di.univr.it/}{ICE Laboratory}}
\faculty{ }

\AtBeginDocument{
	\hypersetup{pdftitle=\ttitle}
	\hypersetup{pdfauthor=\authorname}
	\hypersetup{pdfkeywords=\keywordnames}
}

%	BEGIN DOCUMENT

\begin{document}
	
	\frontmatter % Use roman page numbering style (i, ii, iii, iv...) for the pre-content pages
	\pagestyle{plain} % Default to the plain heading style until the thesis style is called for the body content
	
	%	TITLE PAGE
	\begin{titlepage}
		\begin{center}
			
			\vspace*{.06\textheight}
			{\scshape\LARGE \univname\par}\vspace{1.5cm} % University name
			\textsc{\Large Master Thesis}\\[1.5cm] % Thesis type
			
			\HRule \\[0.4cm] % Horizontal line
			{\LARGE \bfseries \ttitle\par}\vspace{0.4cm} % Thesis title
			\HRule \\[2.5cm] % Horizontal line
			
			\begin{minipage}[t]{0.4\textwidth}
				\begin{flushleft} \large
					\emph{Author:}\\
					\href{mailto:michelepenzo@outlook.it}{\authorname} % Author name
				\end{flushleft}
			\end{minipage}
			\begin{minipage}[t]{0.4\textwidth}
				\begin{flushright} \large
					\emph{Supervisor:} \\
					\href{https://www.di.univr.it/?ent=persona&id=6321&lang=en}{\supname} % Supervisor name
				\end{flushright}
			\end{minipage}\\[3cm]
			
			\vfill
			
			\groupname\\\deptname\\[2cm] % Research group name and department name
			
			\vfill
			% todo scrivere anno accademico
			{\large March 18, 2021}\\[2cm] % Date
			
			\vfill
		\end{center}
	\end{titlepage}
	
	%	ABSTRACT PAGE
	\begin{abstract}
		\addchaptertocentry{\abstractname} % Add the abstract to the table of contents
		\paragraph*{}
		With Industry 4.0 robots have been increasingly adopted in industry. It's therefore necessary to be able to reprogram these robots quickly and easily. Over time, various methodologies have been introduced for reprogramming the tasks that the robot has to perform. In this work, two methodologies for teaching industrial tasks have been compared in all their aspects: \kt{} and \te{} teaching. In addition, a group of users was subjected to an user experiment to find the simplest, most intuitive and comfortable way to perform assembly tasks, considering different characteristics of the users. The work demonstrates how the physical characteristics and the confidence with the pad used for \te{} affect the teaching of a task.\\[2cm]

		\paragraph{}
		Con Industria 4.0 i robot acquisiscono sempre più importanza in ambito industriale, dove è presente una forte necessità di poterli riprogrammare velocemente e in modo semplice. Con il passare del tempo sono state introdotte diverse metodologie per la riprogrammazione dei task che il robot deve compiere. In questo lavoro, sono state confrontate due metodologie per l'insegnamento di task industriali: insegnamento cinestetico e insegnamento tramite teleoperazione. Inoltre, un gruppo di utenti è stato sottoposto a un esperimento per trovare la modalità più semplice, intuitiva e comoda per svolgere task di assemblaggio. Il lavoro dimostra come le caratteristiche fisiche e la confidenza con il pad usato per la teleoperazione influiscono sull'insegnamento dei task.
	\end{abstract}
	
	%	ACKNOWLEDGEMENTS
%	\begin{acknowledgements}
%		\addchaptertocentry{\acknowlabstract edgementname} % Add the acknowledgements to the table of contents
%		\ldots
%	\end{acknowledgements}
	
	
	%	LIST OF CONTENTS
	\tableofcontents 
	\listoffigures
	\listoftables	

	%	ABBREVIATIONS
	\begin{abbreviations}{ll} % Include a list of abbreviations (a table of two columns)
	
	\textbf{LBR} & \textbf{L}eicht \textbf{B}au \textbf{R}oboter\\
	\textbf{IIWA}& \textbf{I}ntellingent \textbf{I}ndustrial \textbf{W}ork \textbf{A}ssistant\\
	\textbf{ROS} & \textbf{R}obot \textbf{O}perating \textbf{S}ystem\\
	\textbf{EE}  & \textbf{E}nd \textbf{E}ffector\\
	\textbf{KT}  & \textbf{K}inesthetic \textbf{T}eaching\\
	\textbf{TT}  & \textbf{T}eleoperation \textbf{T}eaching\\
	\textbf{LBU} & \textbf{L}ow \textbf{B}ody \textbf{U}sers\\
	\textbf{TBU} & \textbf{T}ow \textbf{B}ody \textbf{U}sers\\
	\textbf{RP}  & \textbf{R}egular \textbf{P}layers\\
	\textbf{CP}  & \textbf{C}asual \textbf{P}layers\\
%	\textbf{DOF} & \textbf{D}egree \textbf{O}f \textbf{F}reedom\\
%	\textbf{HMI} & \textbf{H}uman \textbf{M}achine \textbf{I}nterface\\
		
	\end{abbreviations}
	
	
	%	THESIS CONTENT - CHAPTERS
	
	\mainmatter % Begin numeric (1,2,3...) page numbering
	\pagestyle{thesis}
	
	% todo controllare il goal della tesi !!
	% Introduction chapter

\chapter{Introduction}
\label{intro}

%----------------------------------------------------------------------------------------

\section{Motivations}
\label{motivations}


\section{Goals}
\label{goals}


\section{Thesis Overview}
\label{thesis-overview}
% perchè tramite pad e non smartpad? 
	% Literature chapter

\chapter{State of the art}
\label{state-of-the-art}
This chapter describes all the papers and articles that are present in literature and that have been treated in this thesis. Starting from robot learning via demonstration and \te{} to conclude with the state of the art regarding assembly task in industry and presenting the studies that have been done in these areas.
%----------------------------------------------------------------------------------------

% why? intro to industry 4.0
% perchè devono essere le fatte cose modulabili
% prendere dalle slide di ICE
% gravity compensation mode and teleoperation


\section{Robot learning}
\label{robot-learning}
The main goal of robot learning is to create a way for program robot in simple way that are suitable to be used by everyday people. Two interactions methods are compared: \textit{\ks{}} and \textit{\te{}}. In the former, the user physically guides the robot and in the latter the use controls the robot with a pad.

A similar work as mine is \cite{robot-learning}, where they use \ks{} and \te{} using a haptic device and they compare the two ways of interaction. They find that \ks{} is faster in terms of giving a single demonstration and the demonstrations are more successful.

Along the lines of the previous one, \cite{know-acquisition} proposes various approaches for gaining knowledge from human demonstrations to perform assembly tasks in a industrial robotic cell. In this work \ks{} and \te{} using wireless joystick are compared for create point to point movements. Unlike the previous work an experiment is done: three tasks with different aspects have been done by some users.

In mine work a \te{} mode has been developed. In \cite{comparing-teleoperation} they compare a number of teleoperations mode, exploring both the number of dimensions of the control input as well as the most intuitive control spaces. This work propose four methodologies to find a way to move the robot in \te{} using a wireless pad. The modalities are based on mapping joints as full joint and reduced joint or based on task space as using full task space or reducing task space. In their case, since their use case was concrete spraying, the best way to implement \te{} was to reduce task space, but from the experiment the best way in other cases was to implement full task space as described in \ref{teleoperation-pad}.

In \cite{robot-learning-industrial}, a framework for robot learning by multiple human demonstrations is introduced. Through the demonstrations, the robot learns the sequence of actions for
an assembly task without the need of pre-programming. Additionally, the robot learns every path as
needed for object manipulation. Moreover the proposed framework copes with changes in the position and orientation of the objects to be manipulated and also provides obstacle avoidance.


\section{Robot in industry}
\label{robot-industry}
% TODO assembly/modulable task industry

In the early $ 2000 $'s, robot used to perform assembly tasks were still too few, especially the robot used to perform assembly tasks. In the age of Industry 4.0, the need to convert factories in \textit{smart factories} introduced the necessity to have modular platforms that can be changed over the time. In this case, manipulator robot as \kuka{} that can be re-programmed are very useful. In my case, after learning from demonstration, I focused on how \textit{collaborative robot} can be used within smart factories and how they can be re-programmed to perform new tasks.

As described in \cite{cobot-overview}, collaborative robots have been increasingly adopted in industries to facilitate human-robot collaboration. In this paper, an overview of collaborative industrial scenarios and programming requirements for cobots to implement effective collaboration are given. The human operator and the cobot share the same workspace to perform manufacturing processes on work pieces. Different definitions of collaborative scenarios and safety measures are given. Always from this paper, a paragraph about learning from demonstration as \ks{} and \te{} is described.

The main goal for robots in industry is to combine the advantages of robots, which enjoy high levels of accuracy, speed and repeatability, with the flexibility of human workers. In \cite{human-robot-collaboration}, all these aspects are treated. The use of collaborative robots as \kuka{} in industrial processes allows that they can be managed through intuitive systems. One of main challenge is safety, it's fundamental prerequisite in the design of products. Some standards are defined and treated very well this paper.


\section{User study in assembly tasks}
\label{user-study}
% TODO cercare altri casi studio?
As described in \ref{goals}, the main goal of this thesis is to compare different ways of robot learning and make a study over different typologies of users. Much research in the area of robot learning has focused on pick and place tasks while demanding assembly tasks received less attention so far.

Mine user study, and the work made in \cite{user-study-ks} focuses in assembly tasks. This paper evaluate the discrepancies between \ks{} and manual assembly in the context of industrial assembly tasks. In particular they conducted this user study with $ 78 $ participants with different qualities. During the experiment they asked to complete four tasks multiple times to evaluate the learning obtained during the various repetitions. They proved that when the same task is repeated multiple times the learning increase every time, and on the contrary the duration from the first to last trial decrease definitely. These observation confirm the ease of the learning attributed to \ks{}.

% altre 10 righe più o meno ....	
%	% Theory chapter

\chapter{Theory overview}
\label{theory}

%----------------------------------------------------------------------------------------

\section{About the robot}
\label{about-the-robot}



\section{Kinematic chains}
\label{objectives}

	\subsection{Types of Kinematic Chains}
	\label{types-kinematic-chains}
	
		\subsubsection{Direct Kinematic}
		\label{direct-kinematic}

		\subsubsection{Inverse Kinematic}
		\label{inverse-kinematic}		
		
	
	\subsection{Denavit–Hartenberg Convention}
	\label{dh}
		
\section{Theory about robot control}
\label{robot-control}		

	\subsection{Postion control}
	\label{position-control}
	
	\subsection{Cartesian impedance control}
	\label{cartesian-impedance-control}
	
	\subsection{Joint impedance control}
	\label{joint-impedance-control}
	
	% The project

\chapter{The project}
\label{project}
This chapter describes the general setup, its components and a small overview on the tools used for develop the project. Finally the project is explained.

%----------------------------------------------------------------------------------------

\section{Setup overview}
The \kuka{} redundant manipulator is programmed using the KUKA's Sunrise Workbench platform and its Java API's. The usage of an open source stack compatible with ROS allows the usage of the robot in a simple way. 

A Sunrise project, containing one or more Robotic Application can be synchronized to the robot cabinet and executed from the SmartPad. 

The \textit{iiwa stack} provide a Robotic Application that can be used with the robot. It establishes a connection to machines connected via Ethernet to the robot cabinet via ROS. The machine, with ROS installed, will be able to send and receive ROS messages to and from the Robotic Application. The messages used in this stack are taken from the messages available in a standard ROS distribution, but there are other custom ones inside the \texttt{iiwa\_msgs} folder. 

With the stack is simple to manipulate the messages received from the robot and send new ones as command to it, using Python script or ROS functionalities already implemented as services, topics, actions.

\begin{figure}[H]
	\centering
	\includegraphics[scale=0.6]{ros_sunrise.png}
	\caption{Robot control via ROS using \textit{iiwa\_stack} and Sunrise OS}
	\label{ros_sunire}
\end{figure}

For further information about \textit{iiwa\_stack} see \cite{iiwa-stack-link} and the related work \cite{iiwa-stack-paper}. Instead for information about Sunrise OS and Workbench see \cite{thesis}.


	\subsection{KUKA Robot Controller}
	The \kuka{} is controlled via the KUKA Robot Controller, also known as the KUKA Sunrise Cabinet. The KRC is responsible for the transmission control inputs as well as the reading the data of the integrated sensors. In our case, for controlling the Kuka we use the Java application providede by the stack that runs into the SmartPad.

	\subsection{Sunrise.Workbench}
	The Sunrise.Workbench is a tool used to program robot applications in Java, which are loaded into and are executed on the KRC. It offers the possibility to control the robot with the following strategies: position control, velocity control, joint and cartesian impedance control.
	It can also can execute the commonly motion patterns as: spline, point-to-point, linear and circular motions.
	
	Since we have a gripper mounted on media flange, there's a task always active in background that provides a method that can be called via \texttt{ros\_service} to open and close the two jaw. We also have another background task for the \textit{rgb} led present on media flange.

	\subsection{Safety configuration}
	A correct thing to do before start the using of the robot is to check and modify the safety configuration loaded by default. In my case:
	\begin{itemize}
		\item I had to set a new value for every joint in robot configuration. I restrict the value about $ 2 $ degrees.
		\item moreover I added a protected workspace. In this case the robot never goes inside the constraint area.  
	\end{itemize}
	
	
\section{Project implementation}
\label{project-implementation}
As described in section~\ref{goals}, the goal of the thesis was to \ldots % TODO

Staring from the goal, the work was divide into two phases:
\begin{enumerate}
	\item create the mode relative to teach by demonstration, 
	\item create a way to tele-operate the robot in a simple way. 
\end{enumerate}
As described in the next sections, in every phase a way to save waypoints and an action on the gripper was implemented. After that the action was captured by the script and was saved in an \texttt{.csv} file. With a dedicated program and that file, all the actions saved into the file can be replicated by the robot. Then the description of the two developed modalities.

	\subsection{Theach by demonstration}
	\label{teach}
	As described in section \ref{kinestetich-teaching}, teach by demonstration or also called \textit{kinestethic teaching} is a way to move the robot in gravity compensation mode. Using the \textit{iiwa\_stack}, the gravity compensation mode was implemented using a joint impendance control mode where for every joint in robot configuration a stiffness and damping value is setted. Stiffness value must be grater than $ 0 $ and it is expressed in $ Nm/rad $, instead damping value must be between $ 0 $ and $ 1 $. After changing the control mode to joint impendance, the robot seems falls. A force contrary to the gravity must be carried out to keep it up.
	
	On the mediaflange, the green button was dedicated for saving actions on gripper and waypoints.
	
	
	\begin{figure}[H]
		\centering
		\includegraphics[scale=0.8]{media_flange.png}
		\caption{The Mediaflange: $ (1) $ led strip, \\$ (2) $ enabling switch, $ (3) $ application button}
		\label{ros_sunire}
	\end{figure}

	% no apertura gripper perchè ci sono le mani di mezzo
	% led
	
	\subsection{Teleoperation}
	\label{teleoperation-pad}
	% full task space + joint space	
			
	% Materials and methods chapter

\chapter{Materials and methods}
\label{intro}

% descrizione del materiale utilizzato, del robot, di tutto ciò che avviene prima del task, la preparazione, come è stato fato ecc --> in questo caso possono essere inserite molto foto che mostrano come il task è stato creato

%----------------------------------------------------------------------------------------

\section{Materials} 
\label{materials}
% experimental setup


\section{Assembly methods}
\label{assembly-methods}
% metodologie per svolgere il task

	% experiments

\chapter{Experiment}
\label{experiment}

In this chapter a complete and detailed description of the experiment is provided, starting with the experimental design that was divided into three phases and described in every step. Later the data collected are explained. Finally a overview about the participants (who they are and their peculiarity) is provided. 

%----------------------------------------------------------------------------------------

\section{Experimental design}
\label{experimental-design}
The experimental design describes the entire flow which is done by every participants of the experiment. It has been divided into three stages for convenience: the first is the phase before the experiment, the second is the experimental phase and the last one is the final phase after the experiment.


\subsection*{Pre-experiment phase}
\label{pre-experiment-phase}
In this phase all the necessary procedures to conduct the experiment correctly are explained. First of all a privacy policy for data collection was signed by the users. After that, a teach phase was carried out by all the users in order to understand how they can move the robot (in \kt{} or \te) and which buttons they need to press to perform the actions. At this stage users were asked to move a Lego block from a pick position to a place one performing a $ 90 $° rotation to teach him how to use rotation feature. This simple pick and place was performed by the users in both the modalities. When the user completed this phase an ascending unique id was assigned to him. Moreover, every user perform the pre-experiment and the experimental phase without seeing other users do the same.

\subsection*{Experimental phase}
\label{experiment-phase}

The experiment consists in two tasks in ascending order of difficulty repeated for three times. We call the repetitions \textit{``trial''} and the tasks were done by the pick and place phases. In figure \ref{fig:setup}, where the robot is positioned, it's possible to see the pick position. Specifically, according to the task to be performed (first or second), the final positions differ in two positions:
\begin{enumerate}
	\item simple interlocking position (fig. \ref{fig:simple_interlocking}),
	\item difficult interlocking position (fig. \ref{fig:difficult_interlocking}).
\end{enumerate}


\begin{figure}[H]
	\centering
	\includegraphics[scale=0.06]{task2.jpg}
	\caption{Simple interlocking}
	\label{fig:simple_interlocking}
\end{figure}

\begin{figure}[H]
	\centering
	\includegraphics[scale=0.06]{task3.jpg}
	\caption{Difficult interlocking}
	\label{fig:difficult_interlocking}
\end{figure}

\begin{figure}[ht]
	\centering
	\includegraphics[scale=0.07]{setup.jpg}
	\caption{Setup overview}
	\label{fig:setup}
\end{figure}

\noindent
After every trial a \textit{vote} to the work done (based on time, number of waypoints and action on gripper) is given to the user. A generic vote is given instead of reporting the time to complete the experiment to the user. This evaluation metric was chosen because, if the times is given, the user will try to perform the next trials in less time because it competes with itself and he risks to complete the tasks incorrectly. All votes that have been given after each trial were noted. Moreover, some comments about the execution of the tasks have been written and reported in the next sections.
After the \textit{first} trial of every task the replay phase based on the waypoints and on the actions on gripper was shown to the users. In this way the user understands where he can improve.\\
Some clarifications are provided to unify all users:
\begin{itemize}
	\item every trial is considered valid, even if a collision by the user or any other mistake that is made by the user compromises the trial,
	\item if there is an error that doesn't depend by the user, the trial is repeated,
	\item the \kuka{} is positioned over the first object that would be manipulated every time that a new trial has to been done by the user.
\end{itemize}


\subsection*{Post-experiment phase}
\label{post-experiment-phase}

At the end of the experimental phase the users were asked to fill out a questionnaire. This questionnaire is divided in three parts: the first asks personal informations as name, age, approximate height and weight and other questions to better understand how to divide users for the result analysis. The second part resumes some questions from the questionnaire of \textit{Nasa TLX} described in \cite{nasa-tlx}, and the last part asks general questions or optional comments about the experiment. 

\section{Data collection}
\label{data-collection}

During the experiment some data were collected from the ROS\texttt{ topic} available with sample rate of $ 10 Hz $. All these data have been saved in \texttt{.csv} files. Those files are related to: sequence of actions performed by the users during the experiment, cartesian pose and wrench, joint position and velocity and finally joint torque and external torque. Moreover, only the waypoints were extracted from the file of the actions ($ user\_x.csv $) and saved into another file.
In figure \ref{fig:collected_data} there is a scheme of the data collected during the experiment. Only the first sub-folder is shown.

\begin{figure}[H]
	\centering
	\begin{minipage}{8cm}
		\dirtree{%
		.1 csv\_files.		
		.2 kt.
		.3 task\_1.
		.4 rep\_1.
		.5 user\_1.csv.
		.5 user\_1\_waypoints.csv.
		.5 user\_1\_wrench.csv.
		.5 user\_1\_pose.csv.
		.5 user\_1\_joint\_pos\_vel.csv.		
		.5 user\_1\_joint\_ext\_tor.csv.		
		.5 {...}.
		.4 rep\_2.
		.5 {...}.
		.4 rep\_3.	    	    
		.5 {...}.	    
		.3 task\_2.
		.4 {...}.
		.3 task\_3.
		.4 {...}.
		.2 tt.
		.3 {...}.
	}	
	\end{minipage}
	\caption{How the collected data are stored}
	\label{fig:collected_data}
\end{figure}

\section{Participants}
\label{participants}

The total number of participants is ten: seven male and three female. All of them are university graduates or students. Their ages are between $ 24 $ and $ 27 $ years old and distributed as in figure \ref{fig:ages}. Half of them never used a robot as \kuka{}. 

\begin{figure}[H]
	\centering
	\includegraphics[scale=0.2]{plot_ages.png}
	\caption{Age distribution of participants}
	\label{fig:ages}
\end{figure}

\noindent
In questionnaire data on physical characteristics as weight and height were collected to see if there were correlations between the use of \kt{} teaching and the physical characteristics of each user. In figures \ref{fig:height} and \ref{fig:weight} are represented the percentages of users divided by weight and height. All the other intervals ($ 10cm $ for height and $ 10kg $ for weight) weren't included because no participants were included in them. 

\begin{figure}[H]
	\centering
	\includegraphics[scale=0.2]{plot_height.png}
	\caption{Height distribution of the participants}
	\label{fig:height}
\end{figure}

\begin{figure}[H]
	\centering
	\includegraphics[scale=0.2]{plot_weight.png}
	\caption{Weight distribution of the participants}
	\label{fig:weight}
\end{figure}

\noindent
To obtain better and diversified results among the participants, the modality in which each user starts with the experiment was diversified. Half of the participants started with \te{}, the others with \kt{}. To understand if users could be divided according to their abilities to use the pad, was asked them how often they use it: in figure \ref{fig:pad} the answers are represented.
We recognize that we can't assume that these participants describe a representative sample of the population but it's a good sample of data. 

\begin{figure}[H]
	\centering
	\includegraphics[scale=0.2]{plot_pad.png}
	\caption{Answers to the question about the confidence with the pad}
	\label{fig:pad}
\end{figure}


	% experiments and result

\chapter{Results analysis}
\label{results-analysis}

In this chapter all the results are presented in an objective way. In these sections some graphs, figures and tables are presented without discussing the results obtained. In the next chapter, a complete explanation of the results is provided.

%----------------------------------------------------------------------------------------

\section{Comments about trials}
\label{comments-about-trials}

During the experiment all the participants were set free to pick up the objects in the order they prefer. All of them use the order as $ \{ $\textit{triangle, trapezoid and circle}$ \} $, but only one decided to pick up the objects in different order for all the trials only in \kt{}. This doesn't affect the data.\\
For simplicity we summarize by saying that with $ \{t1,t2\} $ is indicated the task and with $ \{r1,r2,r3\} $ the trial. As described in \ref{experimental-design}, for task $ \{t1,t2\} $ three trials in both the modalities were done. From these trials some comments was taken and it's possible to report that:
\begin{itemize}
	\item in \textbf{\kt{}} there aren't many additional comments. Only one user, for every trial, took many waypoints even after it was shown to him how the robot replicates the sequence. Moreover sometimes some users wrong in taking the waypoint and opens or closes the gripper, but this not affect the final realization. Finally, an user who knows the environment well always takes a point just above the place position to make the interlocking without problems: this happens only for the second task.
	
	\item in \textbf{\te{}} there were many problems in teaching phase. Out of ten users, seven of them made a collision that is when the robot exerts too much force on a surface and this activates the collision avoidance feature. In figure \ref{fig:pass_fail} successes and failures during teaching phase are shown. We can see that the first and the second column, that are the first trails have a high number of failures.
	In task $ t1 $ and trial $ r3 $ and task $ t2 $ and trial $ r1,r2 $ we can see that there was only a mistake by all the users: this is acceptable. The surprising value is the last one (task $ t2 $ and trial $ r3 $): it would be logical that this value was zero or small because it's the latest trial and the users should have become very familiar with the pad. The users who made these mistakes were only those who started with \te{}. Overall the error rate during the teach phase is $ 16.6\% $: for a better report it's possible to resume the success rate for each trial of all users as evidenced in table \ref{tab:pass}. It was also noted that during the trials sometimes they forgot to save the waypoints and it was remembered to the users to save them for a better replay phase.
	
	\begin{figure}[H]
		\centering
		\includegraphics[scale=0.2]{pass_fail_teleop.png}
		\caption{Failures in \te{} teaching}
		\label{fig:pass_fail}
	\end{figure}
	
	\begin{table}[H]
		\centering
		\begin{tabular}{c||cccccc}
			& \textbf{t1-r1} & \textbf{t1-r2} &\textbf{t1-r3} & \textbf{t2-r1} &\textbf{t2-r2} &\textbf{t2-r3} \\
			\hline \hline
			\textbf{\checkmark}	&$ 70\% $	&$ 80\% $	&$ 90\% $	&$ 90\% $ &$ 90\% $	&$ 80\% $ \\
		\end{tabular}
		\caption{Success rate for each trial of\\ each task in \te{} teaching}
		\label{tab:pass}
	\end{table}

\end{itemize}

\section{Questionnaire results}
\label{questionnaire-results}

As described in section \ref{post-experiment-phase}, the questionnaire was divided into two phases: the first part has already been presented in \ref{participants}, while the second part asked some questions which had to be answered with a value between $ 1 $ and $ 10 $. The questions with the most important results that are presented are:

\begin{itemize}
	\item how much \textbf{physical effort} was required for \te{}/\kt{} teaching?
	
	In figure \ref{fig:physical_effort} we can immediately spot the differences between the two modalities: in \kt{} since the task are performed moving the robot, a force to move it is necessary. The medium value is $ 5 $ which tells us that it's actually not that difficult to move the robot. On the other side, the boxplot relative to \te{} indicates that there's no physical effort for moving the robot.
	
	\begin{figure}[ht]
		\centering
		\includegraphics[scale=0.2]{physical_effort.png}
		\caption{Physical effort in \kt{} and \te{}}
		\label{fig:physical_effort}
	\end{figure}
	
	% todo newpage
	\newpage
	\item how much \textbf{mental effort} was required for \te{}/\kt{} teaching?
	
	In figure \ref{fig:mental_effort} it's possible to immediately spot that the two modalities requires mental effort similar among them, respect to physical effort. We can notice that the medium value for \te{} is higher that the medium one in \kt{}. Obviously in \te{} for moving the robot it's necessary to remember all the buttons and commands that have been taught in teach phase before the experiment, but this values is lowered by the users which have confidence with the pad. %Moreover a greater mental effort is required by the user have to coordinate and understand the reference frame that is used on the robot compared to the one you are used to using.
	
	\begin{figure}[ht]
		\centering
		\includegraphics[scale=0.2]{mental_effort.png}
		\caption{Mental effort in \kt{} and \te{}}
		\label{fig:mental_effort}
	\end{figure}

\end{itemize}
\noindent From the final comments not obligatory for the purposes of the questionnaire, the users said that the \te{} is very interesting and they propose to map a new button on pad to move faster on long movements and another one to be better precise with the rotation of the EE. Instead they propose to add an audio feedback (not only the \textit{rgb} led) after getting the pose or changing the control mode. Moreover an user propose to let him choose the reference frame that is used to move the robot.

\section{Groups}
\label{groups}

From the questionnaire questions and answers is possible to create some different groups based on personal characteristics. Many different factors were analyzed to see if there was a correlation between the various characteristics of the users and the results obtained from the experiment. From the data is possible to determine two possible classifications based on:

\begin{itemize}
	% TODO trovare qualcosa di meglio per dividere i gruppi
	\item \textbf{physical characteristics}: in questionnaire was asked the weight and height keeping a range between $ 10cm $ and $ 10kg $. From the answers is possible to classify two macro groups, both of them composed by five users.
	The first group is composed by the users which height is less or equal than $ 70kg $ and which height it's less or equal than $ 180cm $. This group is  conveniently identified and referenced in the next sections with \smallusers{} that stands for low body users. The other group, identified with \bigusers{} that stands for tall body users, is composed by users with weight grater that $ 70kg $ and height grater than $ 170cm $. There are also two users in this group that are taller than $ 180cm $.	

	\item \textbf{confidence with the pad}: the same reasoning can be applied to this classification. In the questionnaire was asked the level of confidence with the pad, that is with how frequency the pad is used and from the answers two macro groups both of them composed by 5 people have been identified. The users who replied more than once a week and at least once a month were classified as regular players and these users are called \textbf{RP}. The users who replied that they use the pad once a year or they never used it were classified as casual players and they are called \textbf{CP}. Also all the users who have used the pad in the past are casual players.
\end{itemize}

\noindent Moreover, during the experiment phase five users were made to start with the phase related to \kt{} teaching and the other five with \te{} teaching to better divide users into groups for a possible analysis of these factors as well. In chapter \ref{results-discussion} the results obtained starting from the two classifications described above and from the questionnaire are explained.

\section{Data analysis}
\label{data-analysis}

To have groups with better data, one user for each group with an elevate standard deviation calculated for each task over the three trials has been removed.\\
In the following tables time and distance to complete each task in both the modalities are represented as difference between the two groups that were created based on physical characteristics. 
In table \ref{tab:time} the values are described in terms of seconds, instead in table \ref{tab:distance} the values presented are in meters. For both tables the first two columns indicates the groups, in the third one the difference between the \smallusers{} and \bigusers{} group is provided. In the last column the gain between the performances of the two groups is expresses as percentage. The rows represent which task is performed and in which mode. The positive and negative symbol was kept to indicate which of the two groups resulted faster or which one made less distance.
\begin{table}[H]
	\centering
	\begin{tabular}{l||cccc}
		& \smallusers & \bigusers & \textbf{Difference} & \textbf{Gain}\\
		\hline \hline
		\textbf{TT - Task1} & $ 147 $ & $ 154 $ & $ -7 $  & $ -4,5\% $\\
		\textbf{KT - Task1} & $ 70 $  & $ 64 $  & $ +6 $ & $ 9,4\% $\\
		\textbf{TT - Task2} & $ 175 $ & $ 192 $ & $ -17 $ & $ -8,9\% $\\
		\textbf{KT - Task2} & $ 70 $  & $ 62 $  & $ +8 $  & $ 12,3\% $
	\end{tabular}
	\caption{Time difference in seconds between the two groups}
	\label{tab:time}
\end{table}

\begin{table}[H]
	\centering
	\begin{tabular}{l||cccc}
		& \smallusers & \bigusers & \textbf{Difference} & \textbf{Gain}\\
		\hline \hline
		\textbf{TT - Task1} & $ 2,10 $ & $ 2,23 $ & $ -0,13 $ & $ -5,8\% $\\
		\textbf{KT - Task1} & $ 2,54 $ & $ 2,39 $ & $ +0,15 $ & $ 6,3\% $\\
		\textbf{TT - Task2} & $ 1,97 $ & $ 2,34 $ & $ -0,37 $ & $ -15,8\% $\\
		\textbf{KT - Task2} & $ 2,54 $ & $ 2,29 $ & $ +0,25 $ & $ 10,9\% $
	\end{tabular}
	\caption{Distance difference in meters between the two groups}
	\label{tab:distance}
\end{table}

\noindent
The two figures in \ref{fig:time_and_distance_group_t2} indicates the same values of the previous tables. The figure \ref{fig:time_groups_t2} represents graphically the table \ref{tab:time} and the values on $ x $ axis are in seconds. Instead in figure \ref{fig:distance_groups_t2.png} there are the data represented in table \ref{tab:distance} and the values on $ x $ axis are in meters. From these figures is possible to quickly understand the differences between the two groups and the different modalities. Moreover it's possible to understand the correlation between time and distance. Only times and distances for the task $ t1 $ are represented in the these figures because the figures for task $ t2 $ are similar but with the values described in the relevant table.

\begin{figure}[H]
	\begin{subfigure}{1\textwidth}
		\centering
		\includegraphics[scale=0.2]{time_groups_t2.png}
		\caption{Time in seconds for completing task $ t1 $}
		\label{fig:time_groups_t2}
	\end{subfigure}%
	\\
	\begin{subfigure}{1\textwidth}
		\centering
		\includegraphics[scale=0.2]{distance_groups_t2.png}
		\caption{Distance in meters for completing task $ t1 $}
		\label{fig:distance_groups_t2.png}
	\end{subfigure}
	\caption{Time and distance for completing task $ t1 $ \\in both the modalities for each group}
	\label{fig:time_and_distance_group_t2}
\end{figure}

\noindent
From the questionnaire have been collected the data relative to physical and mental effort. These data have been manipulated with division into groups based on physical characteristics. In table \ref{tab:effort} there is the average of the values indicated by the users in questionnaire. The value that was to be indicated were in scale from $ 1 $ to $ 10 $. As the previous tables, on the first and second column there are the two groups. Instead, on the rows there are the efforts for each modality. The first two rows are relative to mental effort, the first one in \te{} and the second one in \kt{}. The third row is relative to physical effort in \te{} and the last one refers to \kt{}. The overall values without the divisions into groups can be seen in figure \ref{fig:mental_effort} for mental effort and in figure \ref{fig:physical_effort} for physical effort.

\begin{table}[H]
	\centering
	\begin{tabular}{l||cc}
		& \smallusers{} & \bigusers{} \\
		\hline \hline
		\textbf{TT - Mental}	& $ 5,6 $ & $ 6 $ \\
		\textbf{KT - Mental}	& $ 5,2 $ & $ 2,8 $ \\
		\textbf{TT - Physical} 	& $ 1,2 $ & $ 1,2 $ \\
		\textbf{KT - Physical} 	& $ 5,4 $ & $ 3,8 $
	\end{tabular}
	\caption{Physical and mental effort in both the modalities}
	\label{tab:effort}
\end{table}

\noindent
As described in the previous chapters many data were collected during the experiment. One of these is the \texttt{cartesian\_wrench}: it's composed of a vector that represent the force along the three spatial directions $ \{ x,y,z \} $. In figure \ref{fig:plot_wrench} is possible to see the differences between the force applied on EE during the experiment, in \te{} the interaction forces are limited instead in \kt{} the forces are high.  In both figures on $ y $ axis there's the force applied on $ z $ and along $ x $ axis there is the time in seconds to complete the trial. In these figures only one trial of the same user is represented. The same graphs, that are similar, can be calculated on forces applied on $ x $ and $ y $. 

\begin{figure}[H]
	\begin{subfigure}{1\textwidth}
		\centering
		\includegraphics[scale=0.2]{wrench_tt.png}
		\caption{\Te{} teaching}
		\label{fig:wrench_tt}
	\end{subfigure}%
	\\
	\begin{subfigure}{1\textwidth}
		\centering
		\includegraphics[scale=0.2]{wrench_kt.png}
		\caption{\Kt{} teaching}
		\label{fig:wrenck_kt}
	\end{subfigure}
	\caption{Difference between wrench on Z \\in \te{} and \kt{}}
	\label{fig:plot_wrench}
\end{figure}

\noindent
In figure \ref{fig:wrench_all} there is another representation of \texttt{cartesian\_wrench} calculated on $ x,y,z $. These figures represent the difference between \kt{} and \te{} teaching between the two groups always divided by physical characteristics. All the boxplots shown in both the figures are divided in three groups (a group for each value of the wrench $\{ x,y,z \}$) and every group has two boxplots: the first one for \smallusers{} group and the second one for \bigusers{} group.
Each boxplot represent the sum of the forces applied on a certain axes of the EE, by all users divided per groups performing all the trials $\{ r1,r2,r3\} $ of each task $\{ t1,t2 \}$.
On $ x $ axis the force applied on EE is represented in $ N $.

\begin{figure}[H]
	\centering
	\begin{subfigure}{1\textwidth}
		\centering
		\includegraphics[scale=0.2]{wrench_all_kt.png}
		\caption{\Te{} teaching}
		\label{fig:wrench_all_kt}
	\end{subfigure}%
	\\
	\begin{subfigure}{1\textwidth}
		\centering
		\includegraphics[scale=0.2]{wrench_all_tt.png}
		\caption{\Kt{} teaching}
		\label{fig:wrench_all_tt}
	\end{subfigure}
	\caption{Wrench on $ x,y,z $ of users grouped by physical \\characteristics during \te{} and \kt{} teaching}
	\label{fig:wrench_all}
\end{figure}

\noindent
The table represented in figure \ref{tab:ratio} represent the total number of waypoints taken by all users grouped by physical characteristics during all trials for each task in different modalities. It's possible to notice that there are two macro columns for the two groups where each of them contains other three columns. For each macro column, the first column contains the number of trials completed by all users of the group. The second column contains the number of waypoints taken by all users during all the trials and the third column is the ratio between the two first columns. Instead the rows represent the modality and the task that is performed.
The minimum number of waypoints to have a task that can be replayed correctly by the \kuka{} is $ 16 $: three points for every objects except only two points for the first and last object. Instead the optimal number is $ 18 $: three points for every object to pick and place.

\begin{table}[H]
	\centering
	\begin{tabular}{l||ccl||ccl}
		& \multicolumn{3}{c||}{\smallusers} & \multicolumn{3}{c}{\bigusers} \\
		\hline \rule{0pt}{2ex}    		
		& \textbf{trials} & \textbf{waypoints} & \textbf{ratio} & \textbf{trials} & \textbf{waypoints} & \textbf{ratio} \\
		\hline \hline \rule{0pt}{3ex}
		\textbf{TT - Task1}	& $ 12 $ & $ 191 $ & $ 15,9 $ & $ 12 $ & $ 225 $ & $ 18,8 $\\
		\textbf{TT - Task2}	& $ 13 $ & $ 225 $ & $ 17,3 $ & $ 13 $ & $ 267 $ & $ 20,5 $\\
		\textbf{KT - Task1}	& $ 15 $ & $ 256 $ & $ 17 $   & $ 14 $ & $ 294 $ & $ 21 $ \\
		\textbf{KT - Task2}	& $ 15 $ & $ 277 $ & $ 18,5 $ & $ 15 $ & $ 298 $ & $ 19,9 $
	\end{tabular}

	\caption{Difference on waypoints taken during the execution of the task} 
	\label{tab:ratio}
\end{table}

\noindent
In figure \ref{fig:waypoints} are shown the waypoints taken during the execution of the task $ t1 $. The red points are referred to \smallusers{} group, instead the blue ones are for \bigusers{} group. On $ x $ and $ y $ axis there is the locations of the points within the space. Each patch of color indicates the position of the shapes, while the $ \square $ mark indicates the position of the cube above each object to facilitate grasping. All waypoints aren't in the center of the object because the position save is the one relative to the center of the media flange mounted on6 the robot.

\begin{figure}[H]
	\centering
	\begin{subfigure}{1\textwidth}
		\centering
		\includegraphics[scale=0.2]{waypoints_kt_t1.png}
		\caption{\Kt{} teaching}
		\label{fig:waypoints_kt_t1}
	\end{subfigure}%
	\\
	\begin{subfigure}{1\textwidth}
		\centering
		\includegraphics[scale=0.2]{waypoints_tt_t1.png}
		\caption{\Te{} teaching}
		\label{fig:waypoints_tt_t1}
	\end{subfigure}
	\caption{Waypoints precision in \te{} and \kt{} teaching}
	\label{fig:waypoints}
\end{figure}
	% results discussion

\chapter{Results discussion}
\label{results-discussion}

In this chapter a through explanation of data, figures and tables extrapolated during the experiments are given. In addition, personal interpretations are given to explain the results obtained.

%----------------------------------------------------------------------------------------

\section{Discussion}
\label{discussion}

The result discussion is done based on groups divided by physical characteristics. These groups allows us to evaluate the difference between \te{} and \kt{}, and moreover allows us to respond to the questions posed in section \ref{goals}. It's necessary to make an initial clarification: the \smallusers{} group is composed by four users which are defined as regular players. This bias was noticed only after the analysis that was made on users by physical characteristics, and will allow us to make important observations as regards the \te{} modality. Moreover, it was also noted in the analysis phase that four users from the \smallusers{} group never used a robot before this experiment.\\ %This section begins with some clarifications about the two phases: in \kt{} teaching only one trial failed, instead in \te{} teaching many trails failed as described in figure \ref{fig:pass_fail}. This happens because the users aren't very sensitive with the pad and in the critical phases they try to do all the teaching phase using position control and they don't change in impedance control.

\noindent
Starting from the data relative to the distance and time to complete the tasks in both the modalities is possible to affirm that the phase relative to \te{} teaching is slower than the phase relative to \kt{} teaching. This aspect is independent of the division into groups. Also, in figure \ref{fig:time_and_distance_group_t2} starting from the division in group we can notice some differences between the \smallusers{} and the \bigusers{} group. The \smallusers{} group is faster than the other one in \te{}, and this is due to the fact that four users in \smallusers{} group are \regular{}. The concept of time to complete the task is related to the distance to complete the same task: so, if an user travel less distance than he is faster. Instead, regarding the \kt{} phase it's possible to notice a substantial difference with respect to the \te{}: controlling the robot in gravity compensation is absolutely faster than using remote control. But in this case, unlike the \te{} the users of the group \bigusers{} are actually faster than the users of the other group. This is due to the physical characteristics of each user. We can therefore say with certainty that the \te{} unifies the differences about physical characteristics and if the confidence with the pad was the same among all the users, even the times could be unified among all. \\

\noindent
From table \ref{tab:effort} is possible to see the differences in physical and mental effort according to  users of the various groups. These values help us to better understand the differences between the two modalities but also to focus on how much personal physical characteristics affect \te{} and \kt{} teaching. For a better explanation the various efforts in the two modalities are divided into four points:
\begin{itemize}
	\item \textbf{Mental effort in \te{}}
	
	The difference between the mental effort ($ 5,6 $ for \smallusers{} and $ 6 $ for \bigusers) of the groups are not high. The \smallusers{} group has a lower value, which implies that the users of this group were more relaxed during the experiment. This is mainly due to the fact that these users have confidence with the pad, although this values should be higher because four users never seen a robot before this experiment. From these values it's possible to affirm that \te{} isn't very mentally tiring or stressful both for those who are familiar with the pad and who do not have it. The slightest difference between the two groups could be eliminated by giving more time for practicing to the users to become confident with the pad. 

	\item \textbf{Mental effort in \kt{}}
	
	As shown in the relative table the differences between the two groups are substantial but always lower than the values of mental effort in \te{}: these values are $ 5,2 $ for \smallusers{} and $ 2,8 $ for \bigusers. This is mainly due to the fact that more than half of the \smallusers{} group never use a robot before this experiment. Instead, four users of \smallusers{} group started with \kt{} teaching so they had to be more mentally relaxed, but this didn't happen.
	Both the values are lower than the values of the mental effort in \te{}: this makes us understand that \kt{} is easier because the user moves the robot manually without having to remember how to move the robot using the analogs. Instead, in \kt{} only one button that as two codes (open/close gripper and save waypoints) is used to perform actions unlike in \te{} where there are more than one button.
	
	\item \textbf{Physical effort in \te{}}
	
	The concept of physical effort in \te{} doesn't make much sense because there isn't interaction with the robot. In fact, the values obtained from the two groups are identical and very low: $ 1,2 $. In figure \ref{fig:wrench_tt} it's possible to immediately spot that the force applied on EE during the experiment are close to zero, therefore null. There is only a little force due to the fact that the EE is not completely balanced. In fact, \te{} puts users at ease because it minimizes the interaction forces between the robot and the user. It's possible to say that \te{} unifies the differences between the physical characteristics.
	
	
	\item \textbf{Physical effort in \kt{}}
	
	One of the most important values, especially for the analysis that was conducted is relative to the physical effort required to move the robot to perform assembly operations. The values obtained from the questionnaire highlight the differences between the two groups: $ 5,4 $ for \smallusers{} and $ 3,8 $ for \bigusers{}. It's possible to notice that the first group believes that \kt{} teaching is more physically tiring than the second group. This is certainly due to differences in personal physical characteristics. Supporting this concept is the figure \ref{fig:wrench_all_kt} that shown the differences between the two groups. In fact, it's possible to notice that the \smallusers{} groups need more force on $ x $ and $ z $ to move the robot, instead the value on the $ y $ axis remains unchanged from the two groups because no significant movements are made. Due to the increased force used to move the robot, it's possible to understand that the users of the group \smallusers{} are more tired that the users of \bigusers{}.\\
	
\end{itemize}

\noindent
% todo
% risposte dei questionari

\section{Pros and cons of KT and TT}
\label{pros-cons}

After the discussion is possible to analyze the pros and cons of the two modalities to try to define the best for the assembly tasks. The concepts that need to be reiterated are those relating to the physical characteristics and confidence with the pad.
In fact, has been noticed how \kt{} teaching is tiring and difficult for people which don't have large physical dimensions.
To overcome this problem, \te{} teaching allows to eliminate the physical differences between users. Moreover, the abilities and skills of users can be exploited in \te{}  if they are confident enough. While for users who are not confident with the pad, these could be trained using simulators. Always in simulation, the environment where the robot is inserted can be recreated in order to make the task without stopping the production line.
Instead, \te{} can also be used for robots that aren't equipped with collaborative features, for example manipulator robots that are inside a protection cage. In fact, the main difference of \kt{} is that it has interaction with the robot unlike the \te{} where the robot can be controlled remotely.\\

\noindent
Starting with the pros of \kt{} teaching 


	% conclusion

\chapter{Conclusion}
\label{conclusion}

%----------------------------------------------------------------------------------------

\section{Conclusions}
\label{conclusions}

The modalities analyzed in this work certainly have substantial differences as regards their basic use. Surely the first difference to highlight is the speed and naturalness with which the tasks can be taught with kinesthetics. Thanks to this simplicity we can say that \kt{} can be used when you want to teach simple tasks, such as pick and place or interlocking tasks where the interlocking margins are high. To the detriment of this, however, it is necessary to take into account the physical characteristics of the end users, since as we have seen in the previous sections, people with high tonnage are certainly facilitated as they are able to be more precise and fast and are able to control the robot. To overcome this problem, \te{} is therefore used, which, as we have seen, unifies the differences between the physical characteristics, introducing however greater slowness but precision as regards the teaching of tasks. We can therefore say that this can be used for difficult tasks that require precision, as it offers the possibility of exploiting more functionalities of the collaborative robot.


%\section{Future works}
%\label{future-works}

	
	%	BIBLIOGRAPHY
	\printbibliography[heading=bibintoc, title=References]
	
\end{document}  


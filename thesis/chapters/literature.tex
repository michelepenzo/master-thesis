% Literature chapter

\chapter{State of the art}
\label{state-of-the-art}
This chapter describes all the papers and articles that are present in literature and that have been treated in this thesis. Starting from robot learning via demonstration and \te{} to conclude with the state of the art regarding assembly task in industry and presenting the studies that have been done in these areas.
%----------------------------------------------------------------------------------------

% why? intro to industry 4.0
% perchè devono essere le fatte cose modulabili
% prendere dalle slide di ICE
% gravity compensation mode and teleoperation


\section{Robot learning}
\label{robot-learning}
The main goal of robot learning is to create a way for program robot in simple way that are suitable to be used by everyday people. Two interactions methods are compared: \textit{\ks{}} and \textit{\te{}}. In the former, the user physically guides the robot and in the latter the use controls the robot with a pad.

A similar work as mine is \cite{robot-learning}, where they use \ks{} and \te{} using a haptic device and they compare the two ways of interaction. They find that \ks{} is faster in terms of giving a single demonstration and the demonstrations are more successful.

Along the lines of the previous one, \cite{know-acquisition} proposes various approaches for gaining knowledge from human demonstrations to perform assembly tasks in a industrial robotic cell. In this work \ks{} and \te{} using wireless joystick are compared for create point to point movements. Unlike the previous work an experiment is done: three tasks with different aspects have been done by some users.

In mine work a \te{} mode has been developed. In \cite{comparing-teleoperation} they compare a number of teleoperations mode, exploring both the number of dimensions of the control input as well as the most intuitive control spaces. This work propose four methodologies to find a way to move the robot in \te{} using a wireless pad. The modalities are based on mapping joints as full joint and reduced joint or based on task space as using full task space or reducing task space. In their case, since their use case was concrete spraying, the best way to implement \te{} was to reduce task space, but from the experiment the best way in other cases was to implement full task space as described in \ref{teleoperation-pad}.

In \cite{robot-learning-industrial}, a framework for robot learning by multiple human demonstrations is introduced. Through the demonstrations, the robot learns the sequence of actions for
an assembly task without the need of pre-programming. Additionally, the robot learns every path as
needed for object manipulation. Moreover the proposed framework copes with changes in the position and orientation of the objects to be manipulated and also provides obstacle avoidance.


\section{Robot in industry}
\label{robot-industry}
% TODO assembly/modulable task industry

In the early $ 2000 $'s, robot used to perform assembly tasks were still too few, especially the robot used to perform assembly tasks. In the age of Industry 4.0, the need to convert factories in \textit{smart factories} introduced the necessity to have modular platforms that can be changed over the time. In this case, manipulator robot as \kuka{} that can be re-programmed are very useful. In my case, after learning from demonstration, I focused on how \textit{collaborative robot} can be used within smart factories and how they can be re-programmed to perform new tasks.

As described in \cite{cobot-overview}, collaborative robots have been increasingly adopted in industries to facilitate human-robot collaboration. In this paper, an overview of collaborative industrial scenarios and programming requirements for cobots to implement effective collaboration are given. The human operator and the cobot share the same workspace to perform manufacturing processes on work pieces. Different definitions of collaborative scenarios and safety measures are given. Always from this paper, a paragraph about learning from demonstration as \ks{} and \te{} is described.

The main goal for robots in industry is to combine the advantages of robots, which enjoy high levels of accuracy, speed and repeatability, with the flexibility of human workers. In \cite{human-robot-collaboration}, all these aspects are treated. The use of collaborative robots as \kuka{} in industrial processes allows that they can be managed through intuitive systems. One of main challenge is safety, it's fundamental prerequisite in the design of products. Some standards are defined and treated very well this paper.


\section{User study in assembly tasks}
\label{user-study}
% TODO cercare altri casi studio?
As described in \ref{goals}, the main goal of this thesis is to compare different ways of robot learning and make a study over different typologies of users. Much research in the area of robot learning has focused on pick and place tasks while demanding assembly tasks received less attention so far.

Mine user study, and the work made in \cite{user-study-ks} focuses in assembly tasks. This paper evaluate the discrepancies between \ks{} and manual assembly in the context of industrial assembly tasks. In particular they conducted this user study with $ 78 $ participants with different qualities. During the experiment they asked to complete four tasks multiple times to evaluate the learning obtained during the various repetitions. They proved that when the same task is repeated multiple times the learning increase every time, and on the contrary the duration from the first to last trial decrease definitely. These observation confirm the ease of the learning attributed to \ks{}.

% altre 10 righe più o meno ....
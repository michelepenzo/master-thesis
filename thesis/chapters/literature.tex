% Literature chapter

\chapter{State of the art}
\label{state-of-the-art}
This chapter describes all the papers and articles that are present in literature and that have been treated in this thesis. Starting from robot learning via demonstration and teleoperation to conclude with the state of the art regarding assembly task in industry and presenting the studies that are that have been done in these areas.
%----------------------------------------------------------------------------------------

% why? intro to industry 4.0
% perchè devono essere le fatte cose modulabili
% prendere dalle slide di ICW


\section{Robot learning from Demonstration}
\label{learning-from-demonstration}
% introduction about loearning from demonstration
% gravity compensation mode and teleoperation
The main goal of robot learning is to create a way for program robot in simple way that are suitable to be used by everyday people. Two interactions methods are compared:\textit{\ks{}} and \textit{\te{}}. In the former, the user physically guides the robot and in the latter the use controls the robot with a pad. A similar work as mine is \cite{robot-learning}, where they use \ks{} and \te{} using a haptic device and they compare the two ways of interaction. They find that \ks{} is faster in terms of giving a single demonstration and the demonstrations are more successful. Along the lines of the previous one, \cite{know-acquisition} proposes various approaches for gaining knowledge from human demonstrations to perform assembly tasks in a industrial robotic cell. In this work \ks{} and \te{} using wireless joystick are compared for create point to point movements. Unlike the previous work an experiment is done: three tasks with different aspects have been done from users.


% comparing alternate modes TODO 
% knowdlage acquisittion throuch human demonstration DONE


\section{Industrial tasks}
\label{industrial-task}
% assembling and modulable tasks

% TODO
% robot learning of industrial assmebly, qua o la?
% suvery on human robot collaboration
% cobot programming

\section{User study}
\label{user-study}

% TODO
% ks teaching a user study 
% siemens example

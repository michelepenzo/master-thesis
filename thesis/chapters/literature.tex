% Literature chapter

\chapter{State of the art}
\label{state-of-the-art}
This chapter describes the papers and articles that have been treated in this thesis. It describes in the first section the works about robot learning via demonstration and \te{}, in the second section the state of the art regarding assembly task in industry. Finally are presented the user studies that have been done in these areas.
%----------------------------------------------------------------------------------------


\section{Robot learning}
\label{robot-learning}
The main goal of robot learning is to create a way for program robot in a simple way that is suitable to be used by everyday people. Two interactions methods are compared: \textit{\kt{}} and \textit{\te{}}. In the first modality the user physically guides the robot and in the second one the user controls the robot with a pad.

A similar work as mine is \cite{robot-learning}, where they use \kt{} and \te{} using a haptic device and they compare the two ways of interaction. They find that \kt{} is faster in terms of giving a single demonstration and the demonstrations are more successful.

Along the lines of the previous one, \cite{know-acquisition} proposes various approaches for gaining knowledge from human demonstrations to perform assembly tasks in a industrial robotic cell. In this work \kt{} and \te{} using wireless joystick are compared for create point to point movements. Therefore, in this work an experiment is done: three tasks with different aspects have been done by some users.

In our work only one \te{} mode has been developed. In \cite{comparing-teleoperation}, different modalities of \te{} are compared, exploring both the number of dimensions of the control input as well as the most intuitive control spaces. This work propose four methodologies to find a way to move the robot in \te{} using a wireless pad. The modalities are based on mapping joints as full joint and reduced joint or based on task space as using full task space or reducing task space. In this case, since the use case was concrete spraying, the best way to implement \te{} was the reduced task space, but from the experiment the best way in other cases were to implement the full task space as described in \ref{teleoperation-pad}.

In \cite{robot-learning-industrial}, a framework for robot learning by multiple human demonstrations is introduced. Through the demonstrations, the robot learns the sequence of actions for
an assembly task without the need of pre-programming. Additionally, the robot learns every path as
needed for object manipulation. Moreover, the proposed framework copes with changes in the position and orientation of the manipulated objects and also provides obstacle avoidance.

\section{Robot in industry}
\label{robot-industry}

In the age of Industry 4.0, the manufacturing industries increasingly demand more flexible and agile productions systems. The need to convert factories in \textit{smart factories} introduced the necessity to have modular platforms that can be changed over the time. Manipulator robot as \kuka{} that can be re-programmed are very useful. They offer a higher level of hardware flexibility, but in order to benefit from this flexibility, the demand for new approaches to operating and programming new tasks is inevitable. Our work focused on how \textit{collaborative robot} can be used within smart factories and how they can be re-programmed to perform new tasks.

As described in \cite{cobot-overview}, collaborative robots have been increasingly adopted in industries to facilitate human-robot collaboration. In this paper, an overview of collaborative industrial scenarios and programming requirements for cobots is given to implement effective collaboration. The human operator and the cobot share the same workspace to perform manufacturing processes on work pieces. Different definitions of collaborative scenarios and safety measures are given. In this paper, a paragraph about learning from demonstration as \kt{} and \te{} is also described.

The main goal for robots in industry is to combine the advantages of robots, which enjoy high levels of accuracy, speed and repeatability, with the flexibility of human workers. In \cite{human-robot-collaboration}, all these aspects are treated. The use of collaborative robots as \kuka{} in industrial processes allows that they can be managed through intuitive systems. One of main challenge is safety, it's fundamental prerequisite in the design of products. Some standards are defined and treated very well this paper.

\section{User study in assembly tasks}
\label{user-study}

As described in \ref{goals}, the main goal of this thesis is to compare different ways of robot learning and make a study over different typologies of users. A lot of research in the area of robot learning has focused on pick and place tasks while demanding assembly tasks has received less attention.

Our user study and the work made in \cite{user-study-ks} focus in assembly tasks. This paper evaluate the discrepancies between \kt{} and manual assembly in the context of industrial assembly tasks. This user study was conducted with $ 78 $ participants with different characteristics. During the experiment they asked to complete four tasks (two peg-in-hole and two DIN rail of different difficulty) multiple times to evaluate the learning obtained during the various repetitions. They proved that when the same task is repeated multiple times the learning increase every time and, on the contrary, the duration from the first to last trial decrease definitely. These observation confirm the ease of the learning attributed to \kt{}.

The work described in \cite{hri-instructing-industrial} presents a human-robot interface based on task level programming and \kt{}. It was assessed by nine people of different robotic experiences. The main purpose was to obtain feedback from various users and to assess how well they comprehended and operated the system. The test consisted of two separate tasks: a simple pick and place and a more advanced peg in hole task. The users performed the task individually. The presented system shown that \kt{} is an intuitive method for robot programming for non-robotics experts.

% todo newpage
\newpage
The work described in \cite{user-study-redundant} doesn't treat assembly tasks, but propose a new interaction scheme combining \kt{} and learning within an integrated system architecture. They evaluate this approach in a user study with fourty-nine industrial workers. The tasks consist in a warm-up phase: movement from left working area to right working area and wire-loop movement. The results shown that the interaction concepts implemented on a \kuka{} are easy to handle for novice users and provide significantly improved performance for the teaching of trajectories in task space.

As described in \cite{nist-metrics}, new technologies in the areas of robotic arms have the potential to accelerate the use of robotics for assembly. Additionally, robotic hands are emerging as a next generation of EE, with advanced force control and manipulation capabilities. On this paper a set of performance metrics, test methods and associated artifacts are being developed to progress the application of these technologies. Another important aspect of performance measurement is the multiple repetitions of the task: the users acquire more confidence with multiple repetitions.
% The project
% TODO inserire schema della repo come in aa
\chapter{The project}
\label{project}
This chapter describes the general setup, its components and a small overview on the tools used for develop the project. Finally the project is explained.

%----------------------------------------------------------------------------------------

\section{Setup overview}
The \kuka{} redundant manipulator is programmed using the KUKA's Sunrise Workbench platform and its Java API's. The usage of an open source stack compatible with ROS (presented in \cite{ros}) allows the usage of the robot in a simple way. 

A Sunrise project, containing one or more Robotic Application can be synchronized to the robot cabinet and executed from the SmartPad. 

The \textit{iiwa stack} provide a Robotic Application that can be used with the robot. It establishes a connection to machines connected via Ethernet to the robot cabinet via ROS. The machine, with ROS installed, will be able to send and receive ROS messages to and from the Robotic Application. The messages used in this stack are taken from the messages available in a standard ROS distribution, but there are other custom ones inside the \texttt{iiwa\_msgs} folder. 

With the stack is simple to manipulate the messages received from the robot and send new ones as command to it, using Python script or ROS functionalities already implemented as services, topics, actions.

\begin{figure}[H]
	\centering
	\includegraphics[scale=0.6]{ros_sunrise.png}
	\caption{Robot control via ROS using \textit{iiwa\_stack} and Sunrise OS}
	\label{ros_sunire}
\end{figure}

For further information about \textit{iiwa\_stack} see \cite{iiwa-stack-link} and the related work \cite{iiwa-stack-paper}. Instead for information about Sunrise OS and Workbench see \cite{thesis}.


	\subsection{KUKA Robot Controller}
	The \kuka{} is controlled via the KUKA Robot Controller, also known as the KUKA Sunrise Cabinet. The KRC is responsible for the transmission control inputs as well as the reading the data of the integrated sensors. In our case, for controlling the Kuka we use the Java application providede by the stack that runs into the SmartPad.

	\subsection{Sunrise.Workbench}
	The Sunrise.Workbench is a tool used to program robot applications in Java, which are loaded into and are executed on the KRC. It offers the possibility to control the robot with the following strategies: position control, velocity control, joint and cartesian impedance control.
	It can also can execute the commonly motion patterns as: spline, point-to-point, linear and circular motions.
	
	Since we have a gripper mounted on media flange, there's a task always active in background that provides a method that can be called via \texttt{ros\_service} to open and close the two jaw. We also have another background task for the \textit{rgb} led present on media flange.

	\subsection{Safety configuration}
	A correct thing to do before start the using of the robot is to check and modify the safety configuration loaded by default. In my case:
	\begin{itemize}
		\item I had to set a new value for every joint in robot configuration. I restrict the value about $ 2 $ degrees.
		\item moreover I added a protected workspace. In this case the robot never goes inside the constraint area.  
	\end{itemize}
	
	
\section{Project implementation}
\label{project-implementation}
As described in section~\ref{goals}, the goal of the thesis was to \ldots % TODO

Staring from the goal, the work was divide into two phases:
\begin{enumerate}
	\item create the mode relative to teach by demonstration, 
	\item create a way to tele-operate the robot in a simple way. 
\end{enumerate}
As described in the next sections, in every phase a way to save waypoints and an action on the gripper was implemented. After that the action was captured by the script and was saved in an \texttt{.csv} file. With a dedicated program and that file, all the actions saved into the file can be replicated by the robot. Then the description of the two developed modalities.

	\subsection{Teach by demonstration}
	\label{teach}
	As described in section \ref{robot-learning}, teach by demonstration or also called \textit{kinestethic teaching} is a way to move the robot in gravity compensation mode. Using the \textit{iiwa\_stack}, the gravity compensation mode was implemented using joint impendance control mode where for every joint in robot configuration a stiffness and damping value is setted. Stiffness value must be grater than $ 0 $ and it is expressed in $ Nm/rad $, instead damping value must be between $ 0 $ and $ 1 $. After changing the control mode to joint impendance, the robot seems falls. A force contrary to the gravity must be carried out to keep it up.
	
	On media flange, the green button ( \circled{3} in figure \ref{media_flange} ) was dedicated for saving commands. When the button is pressed, according to the duration of the pressure an action is recorded:
	\begin{itemize}
		\item one click for save a waypoint;
		\item $ 2 $ seconds pressure for save action on the gripper;
		\item $ 5 $ seconds pressure to exit from teaching program.
	\end{itemize}
	When an action is taken, the led strip ( \circled{1} in figure \ref{media_flange} ) change color: green for waypoints, blue for gripper's actions. For safety, as the robot must be held up with the hands, no closing action are performed on two jays.
	In the next figure the Media Flange mounted on \kuka.
	
	\begin{figure}[H]
		\centering
		\includegraphics[scale=0.6]{media_flange.png}
		\caption{Media flange: \protect\circled{1} led strip, \\\protect\circled{2} enabling switch, \protect\circled{3} application button}
		\label{media_flange}
	\end{figure}

	\subsection{Teach with remote control}
	\label{teleoperation-pad}
	As described in section \ref{teleoperation}, remote control or \textit{teleoperation} indicates operation of a system or robot at a distance. In our case, teleoperation is intended as control the \kuka{} with a \textbf{PlayStation 4 pad} and it was developed in two ways:

	\begin{itemize}
		\item \textit{full task space}: in this modality the robot will move respect to the EE. A movement along $ x $, $ y $ and $ z $ is performed using linear and angular velocity. With linear velocity you will move the pose of the robot in task space, instead with angular you will move the orientation of the EE.
		\item \textit{reduced joint space}: in this other modality not all joints are considered. Some joint have been blocked, you can move only eh $ 1°$, $ 2° $ and $ 6° $ joint. It's also possible change the orientation of the EE using angular velocity.
	\end{itemize}
	Both of them modalities are performed using \texttt{L1} or \texttt{R1} as deadman buttons with right or left analog.	This two modalities of teleoperation were take from \cite{comparing-teleoperation}. In both cases, the buttons used are the same:
	\begin{itemize}
%		\item $ \bigcirc $
		\item $ \times $: close or open gripper, therefore save action and the actual pose;
		\item $ \triangle $: save actual pose;
		\item $ \square $: change to control mode from position to cartesian impedance, and vice versa.
	\end{itemize}

	Using the pad, it's possible to use the internal vibration. Therefore, when an external force grater than a preset value is detected the \textbf{pad} start vibrating. Based on the controller that the user is using, impedance or position control, the force for activation is different.

	\begin{figure}[H]
		\centering
		\includegraphics[scale=0.38]{pad.png}
		\caption{The scheme of ps4 pad: the model used for teleoperation}
		\label{pad}
	\end{figure}	
		
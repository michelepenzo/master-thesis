% conclusion

\chapter{Conclusion}
\label{conclusion}

%----------------------------------------------------------------------------------------

\section{Conclusions}
\label{conclusions}

The modalities analyzed in this work certainly have substantial differences as regards their basic use. Surely the first difference to highlight is the speed and naturalness with which the tasks can be taught with kinesthetics. Thanks to this simplicity we can say that \kt{} can be used when you want to teach simple tasks, such as pick and place or interlocking tasks where the interlocking margins are high. To the detriment of this, however, it is necessary to take into account the physical characteristics of the end users, since as we have seen in the previous sections, people with high tonnage are certainly facilitated as they are able to be more precise and fast and are able to control the robot. To overcome this problem, \te{} is therefore used, which, as we have seen, unifies the differences between the physical characteristics, introducing however greater slowness but precision as regards the teaching of tasks. We can therefore say that this can be used for difficult tasks that require precision, as it offers the possibility of exploiting more functionalities of the collaborative robot.


%\section{Future works}
%\label{future-works}

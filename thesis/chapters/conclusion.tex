% conclusion

\chapter{Conclusion}
\label{conclusion}

This chapter describes if the goals of the thesis described in \ref{goals} were focused. Moreover, it provides an explanation of how, when and why the two modalities should be used. Finally, the last section describes possible future works and some improvements dictated from the users.

%----------------------------------------------------------------------------------------

\section{Conclusions}
\label{conclusions}

From the abstract, the main goal of this thesis was to create a new methodology for teaching industrial assembly tasks. This goal has been achieved, in fact, as described in section \ref{teleoperation-pad}, a new \te{} modality was proposed and developed. The second goal was to submit to a group of users an experiment, necessary to compare the performances and the differences between the new \te{} modality and the default teaching modality: \kt{}. The results obtained and discussed in \ref{results-discussion} shown how the user experiment brings concrete results.
The two modalities analyzed in this work certainly have substantial differences. Surely the first difference to highlight is the speed and naturalness how the tasks can be taught in \kt{} opposed to \te. Thanks to this simplicity \kt{} teaching can be used when you want to teach simple tasks, such as pick and place or interlocking tasks where the interlocking margins are high. To the detriment of this, however, it's necessary to take into account the physical characteristics of the users since, as we have seen in the previous sections, people with high tonnage are certainly facilitated as they are able to be more precise and fast because they are able to control the robot in simpler way.
To overcome this problem, \te{} is therefore used because unifies the differences between the physical characteristics, introducing greater slowness but precision regarding the teaching of tasks. In addition, in \te{} the abilities of the users can be reused and trained using simulators. We can for this reason say that \te{} teaching can be used for difficult tasks that require precision as it offers the possibility of exploiting more functionalities of the collaborative robot.

\section{Future works}
\label{future-works}

From the improvements dictated by the users we can understand how \te{} is an interesting modality for  teaching assembly tasks. It could be improved with some tricks like inserting an audio feedback to indicate when a button was pressed on the pad. In fact, at the moment there is only visual feedback provided by the \textit{rgb} led. Another suggestion from the users was to let them choose the reference frame that they are used to using.
Surely a test with a population with different physical and personal characteristics could be done to show how they affect the two modalities. Moreover, the modality to train users in simulation could be implemented. 
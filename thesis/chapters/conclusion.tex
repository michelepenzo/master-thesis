% conclusion

\chapter{Conclusion}
\label{conclusion}

%----------------------------------------------------------------------------------------

\section{Conclusions}
\label{conclusions}

The modalities analyzed in this work certainly have substantial differences. Surely the first difference to highlight is the speed and naturalness with how the tasks can be taught in \kt{} opposed to \te. Thanks to this simplicity we can say that \kt{} teaching can be used when you want to teach simple tasks, such as pick and place or interlocking tasks where the interlocking margins are high. To the detriment of this, however, it is necessary to take into account the physical characteristics of the users, since as we have seen in the previous sections, people with high tonnage are certainly facilitated as they are able to be more precise and fast because they are able to control the robot in simpler way.
To overcome this problem, \te{} is therefore used because unifies the differences between the physical characteristics, introducing however greater slowness but precision as regards the teaching of tasks. We can therefore say that this can be used for difficult tasks that require precision, as it offers the possibility of exploiting more functionalities of the collaborative robot.


%\section{Future works}
%\label{future-works}
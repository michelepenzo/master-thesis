% experiments

\chapter{Experiment}
\label{experiment}
% descrizione vera e propria dell'esperimento
% come viene misurato il tempo di contatto, wrench o durata? ...
% cosa viene registrato...
% sample rate

%----------------------------------------------------------------------------------------

\section{Participants}
\label{participants}
% caratteristiche di chi lo fa, quanto ne sanno di robotica, età
% overview generale dei partecipanti


\section{Assembly methods}
\label{assembly-methods}
% metodologie per svolgere il task? quale è la modalità migliore per svolgerli?
% cosa deve fare, come viene fatto?


\section{Assembly tasks}
\label{assembly-tasks}
% quali task? in che cosa consistono i task? 1, 2, 3 ...
% in questo caso possono essere inserite molto foto che mostrano come il task è stato creato


% experiments

\chapter{Experiment}
\label{experiment}

In this chapter a complete description of the experiment in detail is provided. Starting from an overview about the participants: who they are and their peculiarity. In section \ref{experimental-design} the experimental design was divided into phases and described in every step. Finally was described which data were collected.

%----------------------------------------------------------------------------------------

\section{Participants}
\label{participants}
% caratteristiche di chi lo fa, quanto ne sanno di robotica, età
% overview generale dei partecipanti
% quello che viene tirato fuori dai questionari

\section{Experimental design}
\label{experimental-design}
The experimental design describes the entire flow which is done by every participants of the experiment. It has been divided into 3 stages for convenience: the first is the phase before the experiment, the second the experimental phase and the last one the final phase after the experiment.


\subsection*{Pre-experiment phase}
\label{pre-experiment-phase}
In this phase all the procedures necessary to conduct the experiment correctly. First of all a privacy policy for data collection was signed by the users. After that a teach phase was carried out by all the users in order tu understand how they can be move the robot (\kt{} and \te) and which buttons they need to press to perform the actions. At this stage users are asked to move a Lego block from a pick position to a place one performing a $ 90 $° rotation to teach him how to use rotation feature. This simple pick and place was performed in by the users in both the modalities. If the user complete this phase an unique id is assigned to him. Moreover every user perform the pre-experiment and experiment phase without seeing other users do the same.

\subsection*{Experiment phase}
\label{experiment-phase}

The experiment consists in three subtasks in ascending order of difficulty



Some clarifications are provided to unify all users:
\begin{itemize}
	\item every time that a trial is repeated the \kuka were positioned over the first object that would be manipulated,
	\item every trial is considered valid, even if a collision by the user or any other mistake that is made by the user compromises the trial,
	\item if there is an error that does not depend by the user, the trail was repeated
\end{itemize}



\subsection*{Post-experiment phase}
\label{post-experiment-phase}

% metodologie per svolgere il task? quale è la modalità migliore per svolgerli?
% cosa deve fare, come viene fatto?
% quali task? in che cosa consistono i task? 1, 2, 3 ...
% in questo caso possono essere inserite molto foto che mostrano come il task è stato creato


\section{Data collection}
\label{data-collection}
% come viene misurato il tempo di contatto, wrench o durata? ...
% cosa viene registrato...
% sample rate

% experiments

\chapter{Experiment}
\label{experiment}

In this chapter a complete description of the experiment in detail is provided. Starting with the experimental design that was divided into phases and described in every step. Later is' explained which data was collected . Finally an overview about the participants: who they are and their peculiarity. 

%----------------------------------------------------------------------------------------

\section{Experimental design}
\label{experimental-design}
The experimental design describes the entire flow which is done by every participants of the experiment. It has been divided into 3 stages for convenience: the first is the phase before the experiment, the second is the experimental phase and the last one the final phase after the experiment.


\subsection*{Pre-experiment phase}
\label{pre-experiment-phase}
In this phase all the procedures necessary to conduct the experiment correctly are explained. First of all a privacy policy for data collection was signed by the users. After that a teach phase was carried out by all the users in order to understand how they can move the robot (\kt{} and \te) and which buttons they need to press to perform the actions. At this stage users are asked to move a Lego block from a pick position to a place one performing a $ 90 $° rotation to teach him how to use rotation feature. This simple pick and place was performed in by the users in both the modalities. If the user complete this phase an ascending unique id is assigned to him. Moreover every user perform the pre-experiment and experiment phase without seeing other users do the same.

\subsection*{Experiment phase}
\label{experiment-phase}

The experiment consists in two tasks in ascending order of difficulty repeated for \rep{} times. We call the repetitions \textit{``trial''} and the tasks are done by the pick phase from the same position (fig. \ref{fig:pick_pose}) for all the tasks and places in different positions. Specifically, according to the task to be performed (first, second) the final positions differ in positioning in:
\begin{enumerate}
	\item simple interlocking position (fig. \ref{fig:simple_interlocking});
	\item difficult interlocking position (fig. \ref{fig:difficult_interlocking}).
\end{enumerate}


\begin{figure}[H]
	\centering
	\includegraphics[scale=0.06]{task2.jpg}
	\caption{Simple interlocking}
	\label{fig:simple_interlocking}
\end{figure}

\begin{figure}[H]
	\centering
	\includegraphics[scale=0.06]{task3.jpg}
	\caption{Difficult interlocking}
	\label{fig:difficult_interlocking}
\end{figure}

\begin{figure}[ht]
	\centering
	\includegraphics[scale=0.07,angle=-90]{setup.jpg}
	\caption{Setup overview}
	\label{fig:setup}
\end{figure}

After every trial a \textit{vote} to the work done (based on time, number of waypoints and action on gripper) is given to the user. A vote is given because if the time to complete the experiment is given to the user he tries to perform the next trials in the minor time because it competes with itself and he risks to complete tasks incorrectly. All votes that are given after each trial are noted.
After the \textit{first} trial of every sub-task the replay phase based on the waypoints and actions on gripper is shown to the users. In this way the user understands where he can improve.

Some clarifications are provided to unify all users:
\begin{itemize}
	\item every trial is considered valid, even if a collision by the user or any other mistake that is made by the user compromises the trial;
	\item if there is an error that does not depend by the user, the trail was repeated;
	\item every time that a trial is repeated the \kuka{} were positioned over the first object that would be manipulated.
\end{itemize}


\subsection*{Post-experiment phase}
\label{post-experiment-phase}
At the end of the experimental phase an evaluation questionnaire was filled by the users. This questionnaire is divided in three parts: the first asks personal information as name, age, approximate height and weight and other questions to better understand how to divide users for the result analysis . The second part resumes some questions from the questionnaire of \textit{Nasa TLX} described in \cite{nasa-tlx}, and the last part asks general questions about the experiment. \\

\noindent The questionnaire is available at: \href{https://forms.gle/MpqMr7hWY5KoF7gj6}{https://forms.gle/MpqMr7hWY5KoF7gj6}


\section{Data collection}
\label{data-collection}

During the experiment some data was collected from the ROS \texttt{topic} available with sample rate of $ 10 Hz $ and saved in \texttt{.csv} files. Those files are related to: actions from the user during experiment, cartesian pose and wrench, joint position and velocity and finally joint torque and external torque. Moreover from the file of the actions only the waypoints are extracted and saved on another file.
In figure \ref{fig:collected_data} a scheme of the data collected during the experiment. Only the first sub-folder is shown.

\begin{figure}[H]
	\centering
	\begin{minipage}{8cm}
		\dirtree{%
		.1 csv\_files.		
		.2 kt.
		.3 task\_1.
		.4 rep\_1.
		.5 user\_1.csv.
		.5 user\_1\_waypoints.csv.
		.5 user\_1\_wrench.csv.
		.5 user\_1\_pose.csv.
		.5 user\_1\_joint\_pos\_vel.csv.		
		.5 user\_1\_joint\_ext\_tor.csv.		
		.5 {...}.
		.4 rep\_2.
		.5 {...}.
		.4 rep\_3.	    	    
		.5 {...}.	    
		.3 task\_2.
		.4 {...}.
		.3 task\_3.
		.4 {...}.
		.2 teleop.
		.3 {...}.
	}	
	\end{minipage}
	\caption{How the collected data are stored}
	\label{fig:collected_data}
\end{figure}

\section{Participants}
\label{participants}
The total number of participants is ten: seven male and three female. All of them are university graduates or students and between $ 24 $ and $ 27 $ years old and distributed as in figure \ref{fig:ages}. Half of them never used a robot as our \kuka{}. 

\begin{figure}[H]
	\centering
	\includegraphics[scale=0.2]{plot_ages.png}
	\caption{Age distribution of participants}
	\label{fig:ages}
\end{figure}

In questionnaire data on physical characteristics as wight and height were collected to see if there are any correlations between the use of \kt{} and the physical characteristics of each user. In figures \ref{fig:height} and \ref{fig:weight} all the other intervals ($ 10cm $ for height and $ 10kg $ for weight) were not included because no participant was included in them. 

\begin{figure}[H]
	\centering
	\includegraphics[scale=0.2]{plot_height.png}
	\caption{Height distribution of the participants}
	\label{fig:height}
\end{figure}

\begin{figure}[H]
	\centering
	\includegraphics[scale=0.2]{plot_weight.png}
	\caption{Weight distribution of the participants}
	\label{fig:weight}
\end{figure}

To obtain better and diversified results among the participants, the modality from which one starts with the experiment is diversified. Half of the participants start with \te{}, the other with \kt{}. To understand if users could be divided according to their skill in using the pad they were asked at what frequency they use it: in figure \ref{fig:pad} the answers.

\begin{figure}[H]
	\centering
	\includegraphics[scale=0.2]{plot_pad.png}
	\caption{Answers to the question to understand the confidence with the pad}
	\label{fig:pad}
\end{figure}

We recognize that we cannot assume that these participants describe a representative sample of the population but it's a good sample of data. 

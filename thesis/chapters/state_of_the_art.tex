% Literature chapter

\chapter{State of the art}
\label{state-of-the-art}
This chapter describes related works that have been considered in this thesis. Section \ref{robot-industry} describes the state of the art regarding industrial assembly tasks programming, then section \ref{robot-learning} describes the works about robot learning via demonstration and \te{}. Finally, the user studies that have been done in these areas are presented.
%----------------------------------------------------------------------------------------

\section{Robot in industry}
\label{robot-industry}

In the age of Industry 4.0, the manufacturing industries increasingly demand more flexible and agile productions systems. The need to convert factories in \textit{smart factories} introduced the necessity to have modular platforms that can be changed over the time. Manipulator cobot as \kuka{} that can be re-programmed are very useful because they offer a higher level of hardware flexibility, but in order to benefit from this flexibility, the demand for new approaches to operating and programming new tasks is inevitable. Our work focused on how \textit{collaborative robot} can be used within smart factories and how they can be re-programmed to perform new tasks.

Collaborative robots have been increasingly adopted in industries to facilitate human-robot collaboration as described in \cite{cobot-overview}. In this paper, an overview of collaborative industrial scenarios and programming requirements for cobots is given to implement effective collaboration. The human operator and the cobot share the same workspace to perform manufacturing processes on work pieces. Different definitions of collaborative scenarios and safety measures are given. In the work, a paragraph about learning from demonstration as \kt{} and \te{} is also described. We advise the interested reader to refer to this work for more details.

The main goal for robots in industry is to combine the advantages of robots, which enjoy high levels of accuracy, speed and repeatability, with the flexibility of human workers. In \cite{human-robot-collaboration}, all these aspects are treated. The use of collaborative robots as \kuka{} in industrial processes allows that they can be managed through intuitive systems. One of the main challenge is safety that is a fundamental prerequisite in the design of these products. Some standards about safety, collaborative modes, user interfaces are well treated in this paper. For a more detailed explanation in all its aspects, refer to the article presented.


\section{Robot learning}
\label{robot-learning}
The main goal of robot learning is to create a way for program robot in a simple way that is suitable to be used by people without specific skills. 
Two interactions methods considered are: \textit{\kt{}} and \textit{\te{}}. The first one, was considered because it is the default and most used modality for teaching tasks. In this modality the user physically guides the robot. The second one was considered because it offers significant advantages: it permits to have a better and more complete control of the robot, and also to control the robot remotely. It's also used in complex contexts (i.e. handling of toxic substances) where the robot manipulation can only be done remotely.

In \cite{robot-learning}, the authors compare two interaction modes for robot programming: \kt{} and \te{} based on haptic device. They find that \kt{} is faster in terms of giving a single demonstration and the demonstrations are more successful, but this modality does not allow to take full advantage of the collaborative functionalities of the robot. In fact, in our work a \te{} modality was developed for have complete access to these functionalities.

Along the lines of the previous one, \cite{know-acquisition} proposes various approaches for gaining knowledge from human demonstrations to perform assembly tasks in an industrial robotic cell. In this work, \kt{} and \te{} based on a wireless joystick are compared for recording point to point movements. 
Furthermore, the authors perform a user  experiment where three tasks with different aspects have been compared.

In \cite{comparing-teleoperation}, different modalities of \te{} are compared, exploring both the number of dimensions of the control input as well as the most intuitive control spaces. This work propose four methodologies to find the best way to control the robot in \te{} using a wireless pad. The modalities are based on mapping joints or task space. The full joint modality consists to map all the joints of the robot using both the analogs and the directional arrows of the pad. The other modality is based on reducing the joints of the robot: only three joints have been considered and mapped on the pad. The modalities based on the task space consists in moving the robot along the three spatial dimension, for full task space modality, and along only two spatial dimension for reduced task space. In this specific case, since the use case was concrete spraying, the best way to implement \te{} was the reduced task space. However the experiment demonstrates that the best way in other cases was to implement the full task space. Therefore, in our work we consider this \te{} mode, as described in \ref{teleoperation-pad}.

%In \cite{robot-learning-industrial}, a framework for robot learning by multiple human demonstrations is introduced. Through the demonstrations, the robot learns the sequence of actions for an assembly task without the need of pre-programming. Additionally, the robot learns every path as needed for object manipulation. Moreover, the proposed framework copes with changes in the position and orientation of the manipulated objects and also provides obstacle avoidance.

\section{User study in assembly tasks}
\label{user-study}

As described in \ref{goals}, the main goal of this thesis is to compare different ways of robot programming and make a study over different typologies of users. A lot of researches in the area of robot learning have focused on pick and place tasks while demanding assembly tasks has received less attention.

The work presented in \cite{user-study-ks} evaluates the discrepancies between \kt{} and manual assembly in the context of industrial assembly tasks. This user study was conducted with $ 78 $ participants with different characteristics. During the experiment they asked to complete four tasks (two peg-in-hole and two DIN rail of different difficulty) multiple times to evaluate the learning obtained during the various repetitions. They proved that when the same task is repeated multiple times the learning increase every time and, on the contrary, the duration from the first to last trial decrease significantly. These observation confirm the ease of the learning attributed to \kt{}.

The work described in \cite{hri-instructing-industrial} presents a human-robot interface based on task level programming and on \kt{}. It was assessed by nine people of different robotic experiences. The main purpose was to obtain feedback from various users and to assess how well they comprehended and operated the system. The test consisted of two separate tasks: a simple pick and place and a more advanced peg in hole task. The users performed the task individually. The presented system shown that \kt{} is an intuitive method for robot programming for non-robotics experts.

The work described in \cite{user-study-redundant} does not treat assembly tasks, but propose a new interaction scheme combining \kt{} and learning within an integrated system architecture. They evaluate this approach in a user study with fourty-nine industrial workers. The tasks consist in a warm-up phase, in a movement from left working area to right working area and in a wire-loop movement. The results shown that the interaction concepts implemented on a \kuka{} are easy to handle for novice users and provide significantly improved performance for the teaching of trajectories in task space.

As described in \cite{nist-metrics}, new technologies in the areas of robotic arms have the potential to accelerate the use of robotics for assembly. Additionally, robotic hands are emerging as a next generation of EE, with advanced force control and manipulation capabilities. On this paper a set of performance metrics, test methods and associated artifacts are being developed to progress the application of these technologies. Another important aspect of performance measurement is the multiple repetitions of the task: the users acquire more confidence with multiple repetitions.

% todo se ho tempo??
%\section{Substantial differences}
%\label{differences}
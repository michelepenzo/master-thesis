% Introduction chapter

\chapter{Introduction}
\label{intro}

In this chapter a complete description of the thesis is provided: starting from the reasons why this work was done and concluding with the thesis goals. Finally a generic overview is provided.\\

The entire work is available on Github: \href{https://github.com/michelepenzo/master-thesis}{github.com/michelepenzo/master-thesis}


%----------------------------------------------------------------------------------------

\section{Motivations}
\label{motivations}

Within the Industry 4.0, robots are increasingly exploited in production plants. With the ambition to introduce robots into assembly lines the request to be able to quickly reconfigure the workspace requires faster modalities for robot reprogramming. The ease of robot programming is becoming more significant then ever and this process involves a higher degree of automation. For example, welding and painting, are already highly automated in the automotive industry.
Instead, demanding assembly tasks which are mainly pick and place or peg into hole tasks are mainly performed manually today. However, many of these tasks are repetitive, requires high forces and they can be constantly changed. To facilitate reprogramming of robots, the new paradigm which is used more frequently is \textit{Pbd} (Programming By Demonstration) and is often used with assistive and collaborative robots that are installed in industrial environments. 

\section{Goals}
\label{goals}

From \textit{PbD} paradigm described in the previous section, a comparison between two modalities was made to find the optimal method for teaching and reprogramming robots for industrial assembly tasks.
Before starting the work some questions and hypotheses can be done. We therefore define some research questions to better center the goal of this thesis:
\begin{itemize}
	\item Which mode between \kt{} and \te{}, described in \ref{teach} and \ref{teleoperation-pad}, is preferred for ease of use?
	\item Which one requires more physical and mental effort?
	\item The two proposed approaches are said to be intuitive, but how much when they are used for assembly tasks in industry?
	\item There is a correlation between physical characteristics of the users and \kt{} teaching?
	\item People who have a lot of familiarity with joypad are better with \te{}?
\end{itemize}

This work, with the experiment conducted, aims to answer these questions by explaining the reasons for favoring one or the other.


\section{Thesis Overview}
\label{thesis-overview}

The remainder of this thesis is organized as follows: in chapter \ref{state-of-the-art} the state of the art about robot learning and robot in assembly are introduced, moreover the user studies conducted on assembly tasks are presented. In chapter \ref{project} an overview about the robot, the configuration used and how the two modalities for teaching assembly tasks were developed is provided. In chapter \ref{materials-methods} an outline of the user study is done. In this chapter the tests that have been made before the experiment, the research on how the performance can be measured and which materials were used for the experiment are explained. Subsequently, in chapter \ref{experiment} how the experiment is conducted, who participated and other details are explained. Instead, in chapter \ref{results-analysis} the collected data are presented in an objective way and in chapter \ref{results-discussion} the results are explained in detail. Finally, the thesis is concluded in chapter \ref{conclusion} with conclusions.
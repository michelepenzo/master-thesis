% Introduction chapter

\chapter{Introduction}
\label{intro}

In this chapter a complete description of the thesis is provided: starting from the motivations of this work and concluding with the thesis objectives. Finally a generic overview about the structure of the thesis, how the chapters divided and what the chapters explain, is provided. 

The entire code of this thesis is freely (AGPL-3.0 license) available on Github:
\begin{center}
	\href{https://github.com/michelepenzo/master-thesis}{github.com/michelepenzo/master-thesis}
\end{center}


%----------------------------------------------------------------------------------------

\section{Motivations}
\label{motivations}
% todo sistemare questa sezione bene!
Within the Industry 4.0, as described in \cite{industry-4}, robots are increasingly exploited in production plants. Especially, collaborative robots such as \kuka{} are widely used within flexible production lines. They are essential for the entire automation process of assembly tasks, because they introduce more collaborative functions such as impedance control which becomes essential for assembly tasks where greater precision is required. With the ambition to introduce robots into flexible assembly lines, the need to quickly reconfigure the workspace requires faster modalities for robot reprogramming. 
The simplification of robot programming is becoming more significant then ever and this process involves a higher degree of automation. For example, as in figure \ref{painting-assembly}, welding and painting are already highly automated in the automotive industry, while demanding assembly tasks which are mainly pick and place or peg into hole tasks are mostly performed manually today. However, many of these tasks are repetitive, requires high forces and they can be constantly changed. To facilitate reprogramming of robots, the new paradigm which is used more frequently is PBD (Programming By Demonstration) and is often used with assistive and collaborative robots that are installed in industrial environments. 

% todo immagine
\begin{figure}[H]
	\begin{subfigure}{.5\textwidth}
		\centering
		\includegraphics[width=1\linewidth]{painting.jpg}
		\caption{Painting in automotive industry}
		\label{fig:painting}
	\end{subfigure}%
	\begin{subfigure}{.5\textwidth}
		\centering
		\includegraphics[width=1\linewidth]{tt_me.png}
		\caption{Gear assembly task}
		\label{fig:assembly}
	\end{subfigure}
	\caption{Painting and assembly tasks}
	\label{fig:painting-assembly}
\end{figure}


\section{Goals}
\label{goals}

From PBD paradigm described in the previous section, a new methodology to fully exploit the potential of cobots is needed. This work propose a new \te{} methodology for teaching assembly tasks using a remote control like a common joypad for console. In addition, a comparison between \kt{} and \te{} was made to evaluate the performances of the new modality and for find the optimal method for teaching and reprogramming robots for industrial assembly tasks. Before starting the work, we formulate some preliminary questions and hypothesis. Therefore we define some research questions to better focus the goal of this thesis:
\begin{itemize}
	\item Which mode between \kt{} and \te{}, described in \ref{teach} and \ref{teleoperation-pad}, is easier to use?
	\item Which one requires more physical and mental effort?
	\item The two proposed approaches are said to be intuitive, but how much when they are applicable for industrial assembly task programming?
	\item Are the physical characteristics of the users impacting on \kt{} teaching?
	\item People who have a lot of familiarity with joypad are better with \te{}?
\end{itemize}

The experiments conducted in this work aim to answer these questions by explaining the reasons for favoring one or the other.


\section{Thesis Overview}
\label{thesis-overview}

The remainder of this thesis is organized as follows: in chapter \ref{state-of-the-art} the state of the art about robot learning and robot in assembly is introduced, moreover previous user studies conducted on assembly tasks are presented. In chapter \ref{project} an overview about the robot, configuration used and how the two modalities for assembly tasks were developed. In chapter \ref{materials-methods}the user study is described in details. In this chapter the tests that have been made before the experiment, the research on how the performance can be measured and which materials were used for the experiment are explained. Subsequently, in chapter \ref{experiment} how the experiment is conducted, who participated and other details are explained. In chapter \ref{results-analysis} the collected data are presented while in chapter \ref{results-discussion} the results are explained and discussed. Finally, the thesis is concluded in chapter \ref{conclusion} with conclusions.
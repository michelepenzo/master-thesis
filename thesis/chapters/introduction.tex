% Introduction chapter

\chapter{Introduction}
\label{intro}
% intro su industria 4.0, riconfigurazione dello spazio, riprogrammmazione di task diversi
%----------------------------------------------------------------------------------------
In this chapter a complete description of the thesis is provided: started from the reasons for the work done and concluding with the thesis goals and overview.\\

\noindent The entire work is available on Github at: \href{https://github.com/michelepenzo/master-thesis}{github.com/michelepenzo/master-thesis}.


\section{Motivations}
\label{motivations}
Within the Industry 4.0 paradigm robots are increasingly exploited in production plants. The request to be able to quickly reconfigure the workspace requires faster modalities for robot re-programming. This process involves a higher degree of automation. For example, welding and painting, are already highly automated in the automotive industry. Instead, demanding assembly tasks which are mainly pick and place or peg into hole tasks are mainly performed manually toady. However, many of these tasks are repetitive and requires high forces but they can be constantly changed. To facilitate re-programming, the new paradigm which is used more frequently is \textit{Pbd} (Programming By Demonstration) and is often used with assistive robots. This paradigm includes a lot of different approaches as described in section \ref{robot-learning}. 

\section{Goals}
\label{goals}
% domande scentifiche in goals. qual'è la modalità migliore per fare teach?
% perchè tramite pad e non smartpad? 

From PbD paradigm described in the previous section, a comparison to find the optimal method for teaching assembly tasks was sought. The two proposed approaches, described in \ref{teach} and \ref{teleoperation-pad} are said to be intuitive.
% comparison to find the optimal

\section{Thesis Overview}
\label{thesis-overview}
The remainder of this thesis is organized as follows: in chapter \ref{state-of-the-art} the state of the art of robot learning, robot in assembly and the user study conducted on assembly tasks are presented. In chapter \ref{project} an overview about the robot, the configuration used and how the two modalities for teaching assembly tasks is provided. In chapter \ref{materials-methods} an outline of the user study is done. In this chapter the tests that have been made before the experiment, the research on how the performance can be measured and which materials are used for the experiment is done. Subsequently, in chapter \ref{experiment} how the experiment is conducted, who participated and other details are explained. Instead, in chapter \ref{results} the results are explained in detail. Finally, the thesis is concluded in chapter \ref{conclusion} with conclusion and future works .


% experiments and result

\chapter{Results}
\label{results}

In this chapter all the results are presented in an objective way: starting with a complete analysis of the data, and concluding with the result discussion. In these sections some graphs, figures and tables are provided for a better description of the results obtained.

% oggettivo, tabelle, grafici, boxplot, spiego a chi sta leggendo cosa ne è venuto fuori --> devo aiutare chi sta leggendo i risultati a capire cosa fare

%----------------------------------------------------------------------------------------


\section{Results analysis}
\label{results-analysis}
This section has been divided into various subsections for a better description of the results.

\subsection{Comments about trials}
\label{comments-about-trials}
During the experiment all the participants were set free to pick up the objects in the order they prefer. All of them use the order as $ \{ $\textit{triangle, trapezoid and circle}$ \} $, but only one decided to pick up the objects in different order for all the trials only in \kt{}. This doesn't affect the data. \\
As described in \ref{experimental-design}, for every task three trials in both the modalities were done. From these trials some comments was taken and it's possible to report that:
\begin{itemize}
	\item in \textbf{\kt{}} there aren't many additional comments. Only one user, for every trial, took many waypoints even after it was shown to him how the robot replicates the sequence. Moreover sometimes some users wrong in taking the waypoint and opens or closes the gripper, but this not affect the final realization. Finally one user, the most experienced in the use of robots who knows the environment well, always takes a point just above the place position to make the interlocking without problems: this happens only for the second task. No trial had any problems therefore the success rate is $ 100\% $ and all the replay phase shown after the first trial were done correctly by the robot.
	
	\item in \textbf{\te{}} there were many problems in teaching phase. Almost all users, except one, made a collision but how was described in the . This happens because they aren't very sensitive with the pad and they try to do all the teaching phase using position control and not the impedance control. Another motivation is because they are used to using another reference frame when they use the pad. In figure \ref{fig:pass_fail} pass and fail during teaching phase are shown. With $ \{t1,t2\} $ is indicated the task and with $ \{r1,r2,r3\} $ which trial. We can see that the first and the second column, that are the first trails have a high number of failures. This is due to the fact that users have not become confident with the pad. In the task $ t2 $ and trial $ r1,r2 $ we can see that there was only a mistake by all the users: this is acceptable. The surprising value is the last one (task $ t2 $ and trial $ r3 $): it would be logical that this value was zero or small because it's the latest trial and the users should have become very familiar with the pad. The users who made these mistakes were only those who started with \te{}: it's possible say that they were tired and wanted to pass to \kt{} phase because it is less long.
	
	\begin{figure}[ht]
		\centering
		\includegraphics[scale=0.2]{pass_fail_teleop.png}
		\caption{Failures in \te{} teaching}
		\label{fig:pass_fail}
	\end{figure}

	Overall the error rate during the teach phase is $ 18.33\% $: for a better report it's possible to resume the success rate for each trial of all users as evidenced in table \ref{tab:pass}. It was also noted that during the trials they forgot to save the waypoints: sometimes was remembered to the users to save them for a better replay phase. Regarding the replay phase shown after the first trail, if the teaching phase was passed all the replay were done correctly by the robot.
	\begin{table}[ht]
		\centering
		\begin{tabular}{l||llllll}
			& \textbf{t1-r1} & \textbf{t1-r2} &\textbf{t1-r3} & \textbf{t2-r1} &\textbf{t2-r2} &\textbf{t2-r3} \\
			\hline\hline	
			\textbf{\checkmark}	&$ 70\% $	&$ 70\% $	&$ 90\% $	&$ 90\% $ &$ 90\% $	&$ 80\% $ \\
		\end{tabular}
		\caption{Success rate for each trial of\\ each task in \te{} teaching}
		\label{tab:pass}
	\end{table}

\end{itemize}

\subsection{Questionnaire results}
\label{questionnaire-results}
% trovare correlazioni tra la stazza e la difficolta per fare kt, oppure età, gamer ecc...

The questionnaire was divided into two phases as described in section \ref{post-experiment-phase}, the first part has already been presented in \ref{participants}. The second part instead asked mainly some questions which had to be given a score between $ 1 $ and $ 10 $. The questions with the sensible results to presents are:

\begin{itemize}
	\item how much \textbf{physical effort} was required for \te{}/\kt{}?
	
	In figure \ref{fig:physical_effort} we can immediately spot the differences between the two modalities: in \kt{} since the task are performed moving the robot a force to move it is necessary. The medium value is $ 5 $ which tells us that it's actually not that difficult to move the robot. On the other side, the boxplot relative to \te{} indicates that there's no physical effort for moving the robot.
	
	\begin{figure}[ht]
		\centering
		\includegraphics[scale=0.25]{physical_effort.png}
		\caption{Physical effort in \kt{} and \te{}}
		\label{fig:physical_effort}
	\end{figure}

	\item how much \textbf{mental effort} was required for \te{}/\kt{}?
	
	In figure \ref{fig:mental_effort} it's possible to immediately spot that the two modalities requires mental effort similar among them, respect to physical effort. We can notice that the medium value for \te{} is higher that the medium one in \kt{}. Obviously in \te{} for moving the robot it's necessary to remember all the buttons and commands that have been taught in teach phase before the experiment, but this values is lowered by the users which have confidence with the pad. Moreover a greater mental effort is required because the user have to coordinate and understand the reference frame that is used on the robot compared to the one you are used to using.
	
	\begin{figure}[ht]
		\centering
		\includegraphics[scale=0.25]{mental_effort.png}
		\caption{Mental effort in \kt{} and \te{}}
		\label{fig:mental_effort}
	\end{figure}
	
%	\item How \textbf{irritated/stressed} did you feel during the experiment?
%
%	
%
%	\begin{figure}[H]
%		\centering
%		\includegraphics[scale=0.2]{stress_boxplot.png}
%		\caption{Stress and irritation during the experiment}
%		\label{fig:stress}
%	\end{figure}
	
\end{itemize}

\subsection{Groups}
From the questionnaire questions and answers is possible to create some different groups based on personal physical characteristics. Many different factors were analyzed to see if there was a correlation between the various characteristics of the users and the results obtained from the experiment. Two main groups have been highlighted and later presented.

\begin{itemize}
	\item The first one is based on use and confidence with the pad.
	\item The second group is highlighted according to the starting mode defined to do the experiment.
\end{itemize}


% todo
% inserire anche i commenti aggiuntivi che sono stati scritti alla fine

\subsection{Data analysis}
\label{data-analysis}
% grafici sulla forza, durata, velocità e accelerazione6

\section{Result discussion}
\label{result-discussion}

% cosa ho ottenuto, cosa confermano e cosa non confermano (gamer utilizzando meglio il pad, ecc..)
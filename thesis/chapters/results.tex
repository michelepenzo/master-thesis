% experiments and result

\chapter{Results}
\label{results}

In this chapter all the results are presented in an objective way: starting with a complete analysis of the data, and concluding with the result discussion. In these sections some graphs, figures and tables are provided for a better description of the results obtained.

% oggettivo, tabelle, grafici, boxplot, spiego a chi sta leggendo cosa ne è venuto fuori --> devo aiutare chi sta leggendo i risultati a capire cosa fare

%----------------------------------------------------------------------------------------


\section{Results analysis}
\label{results-analysis}

\subsection*{Comments about trials}
As described in \ref{experimental-design}, for every task three trials in both the modalities were done. It's possible to report that:
\begin{itemize}
	\item in \textbf{\kt{}} there aren't many additional comments. Only one user, for every trial, took many waypoints even after it was shown to him how the robot replicates the sequence. Moreover sometimes some users wrong in taking the waypoint and opens or closes the gripper, but this not affect the final realization. Finally one user, the most experienced in the use of robots who knows the environment well, always takes a point just above the place position to make the interlocking without problems: this happens only for the second task.
	
	\item in \textbf{\te{}} there were many problems in teaching phase. Almost all users, except one, made a collision. This happens because they are 
	
\end{itemize}

% dire cosa è andato male e cosa bene --> fare un grafico degli insuccessi

\subsection*{Questionnaire results}
% inserire anche i commenti aggiuntivi che sono stati scritti alla fine



\section{Result discussion}
\label{result-discussion}

% cosa ho ottenuto, cosa confermano e cosa non confermano (gamer utilizzando meglio il pad, ecc..)
% experiments and result

\chapter{Results}
\label{results}

In this chapter all the results are presented in an objective way: starting with a complete analysis of the data, and concluding with the result discussion. In these sections some graphs, figures and tables are provided for a better description of the results obtained.

% oggettivo, tabelle, grafici, boxplot, spiego a chi sta leggendo cosa ne è venuto fuori --> devo aiutare chi sta leggendo i risultati a capire cosa fare

%----------------------------------------------------------------------------------------


\section{Results analysis}
\label{results-analysis}
This section has been divided into various subsections for a better description of the results.


\subsection{Comments about trials}
As described in \ref{experimental-design}, for every task three trials in both the modalities were done. \\
It's possible to report that:
\begin{itemize}
	\item in \textbf{\kt{}} there aren't many additional comments. Only one user, for every trial, took many waypoints even after it was shown to him how the robot replicates the sequence. Moreover sometimes some users wrong in taking the waypoint and opens or closes the gripper, but this not affect the final realization. Finally one user, the most experienced in the use of robots who knows the environment well, always takes a point just above the place position to make the interlocking without problems: this happens only for the second task. No trial had any problems, and all the replay phase shown after the first trial were done correctly by the robot.
	
	\item in \textbf{\te{}} there were many problems in teaching phase. Almost all users, except one, made a collision but how was described in the . This happens because they aren't very sensitive with the pad and they try to do all the teaching phase using position control and not the impedance control. Another motivation is because they are used to using another reference frame when they use the pad. In figure \ref{fig:pass_fail} pass and fail during teaching phase are shown. With $ \{t1,t2\} $ is indicated the task and with $ \{r1,r2,r3\} $ which trial. We can see that the first and the second column, that are the first trails have a high number of failures. This is due to the fact that users have not become confident with the pad. In the task $ t2 $ and trial $ r1,r2 $ we can see that there was only a mistake by all the users: this is acceptable. The surprising value is the last one, in task $ t2 $ and trial $ r3 $: it would be logical that this value was zero or small as possible but it isn't. The users who made these mistakes were only those who started with \te{}: it's possible say that they were tired and wanted to pass to \kt{} phase because it is less long. It's possible to resume these percentages of pass for every trial in table \ref{tab:pass}. Regarding the replay phase shown after the first trail, if the teaching phase was passed all the replay were done correctly by the robot.
	
	\begin{figure}[ht]
		\centering
		\includegraphics[scale=0.2]{pass_fail_teleop.png}
		\caption{Pass and fail in \te{} teaching}
		\label{fig:pass_fail}
	\end{figure}

	\begin{table}[ht]
		\centering
		\begin{tabular}{l||llllll}
			& \textbf{t1-r1} & \textbf{t1-r2} &\textbf{t1-r3} & \textbf{t2-r1} &\textbf{t2-r2} &\textbf{t2-r3} \\
			\hline\hline	
			\textbf{pass}	&$ 80\% $	&$ 50\% $	&$ 100\% $	&$ 90\% $ &$ 90\% $	&$ 70\% $ \\
		\end{tabular}
		\caption{Percentages of pass for every trial of\\ each task in \te{} teaching}
		\label{tab:pass}
	\end{table}

\end{itemize}

\subsection{Questionnaire results}
% inserire anche i commenti aggiuntivi che sono stati scritti alla fine



\section{Result discussion}
\label{result-discussion}

% cosa ho ottenuto, cosa confermano e cosa non confermano (gamer utilizzando meglio il pad, ecc..)
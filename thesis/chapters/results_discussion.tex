% results discussion

\chapter{Results discussion}
\label{results-discussion}

In this chapter a through explanation of data, figures and tables extrapolated during the experiments are given. In addition, personal interpretations are given to explain the results obtained.

%----------------------------------------------------------------------------------------

\section{Discussion}
\label{discussion}

The result discussion is done based on groups divided by physical characteristics. These groups allows us to evaluate the difference between \te{} and \kt{}, and moreover allows us to respond to the questions posed in section \ref{goals}. It's necessary to make an initial clarification: the \smallusers{} group is composed by four users which are defined as regular players. This bias was noticed only after the analysis that was made on users by physical characteristics, and will allow us to make important observations as regards the \te{} modality. Moreover, it was also noted in the analysis phase that four users from the \smallusers{} group never used a robot before this experiment.\\ %This section begins with some clarifications about the two phases: in \kt{} teaching only one trial failed, instead in \te{} teaching many trails failed as described in figure \ref{fig:pass_fail}. This happens because the users aren't very sensitive with the pad and in the critical phases they try to do all the teaching phase using position control and they don't change in impedance control.

\noindent
Starting from the data relative to the distance and time to complete the tasks in both the modalities is possible to affirm that the phase relative to \te{} teaching is slower than the phase relative to \kt{} teaching. This aspect is independent of the division into groups. Also, in figure \ref{fig:time_and_distance_group_t2} starting from the division in group we can notice some differences between the \smallusers{} and the \bigusers{} group. The \smallusers{} group is faster than the other one in \te{}, and this is due to the fact that four users in \smallusers{} group are \regular{}. The concept of time to complete the task is related to the distance to complete the same task: so, if an user travel less distance than he is faster. Instead, regarding the \kt{} phase it's possible to notice a substantial difference with respect to the \te{}: controlling the robot in gravity compensation is absolutely faster than using remote control. But in this case, unlike the \te{} the users of the group \bigusers{} are actually faster than the users of the other group. This is due to the physical characteristics of each user. We can therefore say with certainty that the \te{} unifies the differences about physical characteristics and if the confidence with the pad was the same among all the users, even the times could be unified among all. \\

\noindent
From table \ref{tab:effort} is possible to see the differences in physical and mental effort according to  users of the various groups. These values help us to better understand the differences between the two modalities but also to focus on how much personal physical characteristics affect \te{} and \kt{} teaching. For a better explanation the various efforts in the two modalities are divided into four points:
\begin{itemize}
	\item \textbf{Mental effort in \te{}}
	
	The difference between the mental effort ($ 5,6 $ for \smallusers{} and $ 6 $ for \bigusers) of the groups are not high. The \smallusers{} group has a lower value, which implies that the users of this group were more relaxed during the experiment. This is mainly due to the fact that these users have confidence with the pad, although this values should be higher because four users never seen a robot before this experiment. From these values it's possible to affirm that \te{} isn't very mentally tiring or stressful both for those who are familiar with the pad and who do not have it. The slightest difference between the two groups could be eliminated by giving more time for practicing to the users to become confident with the pad. 

	\item \textbf{Mental effort in \kt{}}
	
	As shown in the relative table the differences between the two groups are substantial but always lower than the values of mental effort in \te{}: these values are $ 5,2 $ for \smallusers{} and $ 2,8 $ for \bigusers. This is mainly due to the fact that more than half of the \smallusers{} group never use a robot before this experiment. Instead, four users of \smallusers{} group started with \kt{} teaching so they had to be more mentally relaxed, but this didn't happen.
	Both the values are lower than the values of the mental effort in \te{}: this makes us understand that \kt{} is easier because the user moves the robot manually without having to remember how to move the robot using the analogs. Instead, in \kt{} only one button that as two codes (open/close gripper and save waypoints) is used to perform actions unlike in \te{} where there are more than one button.
	
	\item \textbf{Physical effort in \te{}}
	
	The concept of physical effort in \te{} doesn't make much sense because there isn't interaction with the robot. In fact, the values obtained from the two groups are identical and very low: $ 1,2 $. In figure \ref{fig:wrench_tt} it's possible to immediately spot that the force applied on EE during the experiment are close to zero, therefore null. There is only a little force due to the fact that the EE is not completely balanced. In fact, \te{} puts users at ease because it minimizes the interaction forces between the robot and the user. It's possible to say that \te{} unifies the differences between the physical characteristics.
	
	
	\item \textbf{Physical effort in \kt{}}
	
	One of the most important values, especially for the analysis that was conducted is relative to the physical effort required to move the robot to perform assembly operations. The values obtained from the questionnaire highlight the differences between the two groups: $ 5,4 $ for \smallusers{} and $ 3,8 $ for \bigusers{}. It's possible to notice that the first group believes that \kt{} teaching is more physically tiring than the second group. This is certainly due to differences in personal physical characteristics. Supporting this concept is the figure \ref{fig:wrench_all_kt} that shown the differences between the two groups. In fact, it's possible to notice that the \smallusers{} groups need more force on $ x $ and $ z $ to move the robot, instead the value on the $ y $ axis remains unchanged from the two groups because no significant movements are made. Due to the increased force used to move the robot, it's possible to understand that the users of the group \smallusers{} are more tired that the users of \bigusers{}.\\
	
\end{itemize}

\noindent
% todo
% risposte dei questionari

\section{Pros and cons of KT and TT}
\label{pros-cons}

After the discussion is possible to analyze the pros and cons of the two modalities to try to define the best for the assembly tasks. The concepts that need to be reiterated are those relating to the physical characteristics and confidence with the pad.
In fact, has been noticed how \kt{} teaching is tiring and difficult for people which don't have large physical dimensions.
To overcome this problem, \te{} teaching allows to eliminate the physical differences between users. Moreover, the abilities and skills of users can be exploited in \te{}  if they are confident enough. While for users who are not confident with the pad, these could be trained using simulators. Always in simulation, the environment where the robot is inserted can be recreated in order to make the task without stopping the production line.
Instead, \te{} can also be used for robots that aren't equipped with collaborative features, for example manipulator robots that are inside a protection cage. In fact, the main difference of \kt{} is that it has interaction with the robot unlike the \te{} where the robot can be controlled remotely.\\

\noindent
Starting with the pros of \kt{} teaching 


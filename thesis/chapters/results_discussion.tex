% results discussion

\chapter{Results discussion}
\label{results-discussion}

In this chapter a thorough explanation of data, figures and tables extrapolated during the experiments are given. In addition, personal interpretations are given to explain the results obtained.

%----------------------------------------------------------------------------------------

\section{Discussion}
\label{discussion}

The results discussion is done based on groups divided by physical characteristics. These groups allows us to evaluate the difference between \te{} and \kt{} teaching, and moreover allows us to respond to the questions posed in section \ref{goals}. It's necessary to make an initial clarification: the \smallusers{} group is composed by four users which are defined as regular players (\regular). This bias was noticed only after the analysis that was made on users by physical characteristics, and will allow us to make important observations as regards the \te. Moreover, it was also noted in the analysis phase that four users from the \smallusers{} group never used a robot before this experiment.\\

Starting from the data relative to the distance and time to complete the tasks in both the modalities, is possible to affirm that the phase relative to \te{} teaching is slower than the phase relative to \kt{} teaching. This aspect is independent of the division into groups .
Also, in figure \ref{fig:time_and_distance_group_t2} starting from the division in group we can notice some differences between the \smallusers{} and the \bigusers{} group. The \smallusers{} group is faster than the other one in \te{}, and this is due to the fact that four users in \smallusers{} group are \regular{}. The concept of time to complete the task is related to the distance to complete the same task: so, if an user travel less distance than he is faster. Instead, regarding the \kt{} phase it's possible to notice a substantial difference with respect to the \te{}: controlling the robot in gravity compensation is absolutely faster than using remote control even if the skills of the users are clearly different from each other.
Unlike the \te{}, the users of the group \bigusers{} are actually faster than the users of the other group. This is due to the physical characteristics of each user. Related to this concept there is also the force applied on the robot to carry out the movements, but this will be explained later.
We can therefore say with certainty that the \te{} unifies the differences about physical characteristics and if the confidence with the pad was the same among all the users, even the times could be unified among all. \\

% todo
% precisione waypoints
% spiegare il problema delle collisioni --> quale gruppo ne ha fatte di più?
% risposte dei questionari

From table \ref{tab:effort} is possible to see the differences in physical and mental effort according to the users of the various groups. These values help us to better understand the differences between the two modalities but also to focus on how much personal physical characteristics affect \te{} and \kt{} teaching. For a better explanation the various efforts in the two modalities have been divided into four points:
\begin{itemize}
	% ===== DA QUI IN POI =====
	\item \textbf{Mental effort in \te{}}
	
	The difference between the mental effort ($ 5,6 $ for \smallusers{} and $ 6 $ for \bigusers) of the groups is not relevant. The \smallusers{} group has a lower value, which implies that the users of this group were more relaxed during the experiment. This is mainly due to the fact that these users have confidence with the pad, although this values should be higher because four users never seen a robot before this experiment. From these values it's possible to affirm that \te{} isn't very mentally tiring or stressful both for those who are familiar with the pad and who do not have it. Instead, \te{} puts people at ease because there is no interaction with the robot. The slightest difference between the two groups could be eliminated by giving more time for practicing to the users to become confident with the pad. 

	\item \textbf{Mental effort in \kt{}}
	
	As shown in the relative table the differences between the two groups are substantial but always lower than the values of mental effort in \te{}: these values are $ 5,2 $ for \smallusers{} and $ 2,8 $ for \bigusers. This is mainly due to the fact that more than half of the \smallusers{} group never use a robot before this experiment. Instead, four users of \smallusers{} group started with \kt{} teaching so they had to be more mentally relaxed, but this didn't happen.
	Both the values are lower than the values of the mental effort in \te{}: this makes us understand that \kt{} is easier because the user moves the robot manually without having to remember how to move the robot using the analogs. Instead, in \kt{} only one button that as two codes (open/close gripper and save waypoints) is used to perform actions unlike in \te{} where there are more than one button. \Kt{} teaching is rated as easy by users who are familiar with robots.	
	
	\item \textbf{Physical effort in \te{}}
	
	The concept of physical effort in \te{} doesn't make much sense because there isn't interaction with the robot. In fact, the values obtained from the two groups are identical and very low: $ 1,2 $. In figure \ref{fig:wrench_tt} it's possible to immediately spot that the force applied on EE during the experiment are close to zero, therefore null. Also from figure \ref{fig:wrench_all_tt} it's possible to see how the values for the two groups are always the same. There is only a little force due to the fact that the EE is not completely balanced. In fact, \te{} puts users at ease because it minimizes the interaction forces between the robot and the user. It's possible to say that \te{} unifies the differences between the physical characteristics.
	
	
	\item \textbf{Physical effort in \kt{}}
	
	One of the most important value, especially for the analysis that was conducted, is relative to the physical effort required to move the robot to perform assembly operations.
	It's immediately noticed how the effort to move the robot unlike the \te{} is completely different, in fact in figure \ref{fig:wrenck_kt} is' seen how the values of the graph oscillate with greater difference that those in figure \ref{fig:wrench_tt}. This means that there's physical interaction to move the robot.
	The values obtained from the questionnaire highlight the differences between the two groups: $ 5,4 $ for \smallusers{} and $ 3,8 $ for \bigusers{}.
	It's possible to notice that the first group believes that \kt{} teaching is more physically tiring than the second group. This is certainly due to differences in personal physical characteristics. Supporting this concept is the figure \ref{fig:wrench_all_kt} that shown the differences between the two groups. In fact, it's possible to notice that the \smallusers{} groups need more force on $ x $ and $ z $ to move the robot, instead the value on the $ y $ axis remains unchanged from the two groups because no significant movements are made on this axis. Due to the increased force used to move the robot, it's possible to understand that the users of the group \smallusers{} are more tired that the users of \bigusers{}.\\
	
\end{itemize}

\section{Pros and cons of KT and TT}
\label{pros-cons}
 
Starting from the discussion of the results is possible to analyze the pros and cons of the two modalities to try to define the best one for assembly tasks. The concepts that need to be reiterated are those relating to the physical characteristics and confidence with the pad. In fact, has been noticed how \kt{} teaching is tiring and difficult for people which don't have high physical size.
To overcome this problem, \te{} teaching allows to eliminate the physical differences between users.
Instead, \te{} can also be used for robots that aren't equipped with collaborative features, for example manipulator robots that are inside a protection cage. In fact, in \kt{} teaching an user has physically interaction with the robot unlike in \te{} teaching where the robot can be controlled remotely.\\\

As first pros of \kt{} teaching, we can certainly talk about its ease and intuitiveness in its use.
The fact of having only one button which allows to perform the task by making all the operations in an easy and quick way. Instead, on the other side in this modality since there is only button no other advanced functionalities can be applied. For example, the change from position control in impedance or force control cannot be performed while the user is teaching the robot how perform the task. In fact, the main cons is that only position control can be used for replaying the sequence of waypoints and action on gripper performed by the user.
If this modality is considered intuitive and fast by  all users just for having one button, the teaching phase cannot be much improved in terms of time with multiple repetitions.
There always be a time limit, this depends on the fact that the various actions that must be taught by the user by pressing the button on the robot, are coded in terms of seconds.
Another cons is that in this mode, as described in table \ref{tab:ratio}, the ratio of waypoints taken during the teaching phase is almost always higher than the ratio of points taken in \te{} teaching. So, if there are more waypoints the replay phase will be slower.\\


Regarding the pros and cons of \te{} teaching, we must say that the possibility to change from position control to impedance control during the teaching phase is one of the main advantages. This allows us to take full advantages from collaborative features. Instead, with remote control the ratio of the waypoints taken during the experiment is in line with the minimum and optimal number of waypoints that must be taken for a correctly replay of the trial. Therefore this allows to have a fast and minimized replay phase. 
Always related to the concept of confidence with the pad, it's well known that this methodology is more difficult than \kt{} teaching because users need to remember more buttons.
The biggest cons is that a lot of collision are made when an inexperienced user want to teach the robot how to do tasks. For users who are not confident with the pad, the abilities of these users could be trained using simulators. While, for the users with experience and confident enough with the pad the abilities and skills can be exploited in \te{}. Always in simulation, the environment where the robot is inserted can be recreated in order to make the task without stopping the production line.
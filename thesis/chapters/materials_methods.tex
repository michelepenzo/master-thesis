% materials and methods

\chapter{Materials and Methods}
\label{materials-methods}

This chapter described the hardware and software setup: the components of the robot, how the robot can be programmed, the tools used to develop the project and which open source stack is used. Finally, the two proposed teaching modalities are explained.

%----------------------------------------------------------------------------------------

\section{Hardware setup}
\label{about-the-robot}

The \kuka{} is a robot with collaborative features. It has a 14 kg payload and an action range from $ 800mm $ to $ 820 mm $. It's one of the latest robotic innovation, with the ability to work in a shared workspace with humans. It has built in sensors and soft edges that make it safe for human collaboration and the ability to detect movement and touch all over. Moreover it has a media flange with integrated wiring that makes it possible to easily interchange different tools. When a tool is attached, it is automatically considered in the kinematic chain. In our case, the tool mounted on the media flange is a \textit{Gimatic MPLM3240}. It is a electric parallel gripper with two self-centering jaw and a total gripping force of $ 210 N $. The Gimatic tool and the media flange are highlighted in figure \ref{kuka-gimatic}.\\

\begin{figure}[H]
	\centering
	\begin{tikzpicture}
	\node(a){\includegraphics[scale=0.18]{kuka-gimatic.jpg} };
	\node at(a.center)[draw, red,line width=2pt,circle, minimum width=80pt, minimum height=10pt,rotate=-28, yshift=12pt, xshift=6pt]{};
	\end{tikzpicture}   
	\caption{Gimatic tool mounted on the media\\ flange of the \kuka{}}
	\label{kuka-gimatic}
\end{figure}

\newpage
\section{Software setup}
The \kuka{} is programmed using the KUKA's Sunrise Workbench platform and its Java API's. The \kuka{} is controlled via the KUKA Robot Controller, also known as the KUKA Sunrise Cabinet. The KRC is responsible for the control inputs transmission as well as reading the data of the integrated sensors.
The Sunrise Workbench is a tool used to program robot applications in Java, which are loaded and executed by the KRC. It offers the possibility to control the robot with the following strategies: position control, velocity control, joint and cartesian impedance control.
It can also execute the commonly motion patterns as: spline, point-to-point or linear motions.

An open source stack compatible with ROS (presented in \cite{ros}) allows the use of the robot in a simple way. A Sunrise project, containing one or more Robotic Application, can be synchronized to the robot cabinet and executed from the SmartPad.
The open source stack, called by developers \textit{iiwa\_stack}, provides a Robotic Application that can be used with the robot that runs over KRC. It establishes a point to point connection via ROS to the machine connected via Ethernet to the robot cabinet. The machine, with ROS installed, will be able to send and receive ROS messages to and from the Robotic Application, as represented in figure \ref{ros_sunire}. The messages used in this stack are taken from the messages available in a standard ROS distribution, but there are other custom ones inside the \texttt{iiwa\_msgs} folder. 

With the stack is simple to manipulate the messages received from the robot and send new ones as command to it. This is possible using Python, C++ or console and the ROS functionalities already implemented as \texttt{service}, \texttt{topic} and \texttt{action}.\\\\

\begin{figure}[H]
	\centering
	\includegraphics[scale=0.2]{ros_sunrise.png}
	\caption{Scheme of robot control using ROS}
	\label{ros_sunire}
\end{figure}

Since we have a gripper mounted on media flange (explained better in \ref{about-the-robot}), there's a task always active in background that provides a method that can be called via \texttt{ros\_service}. This service allows us to open and close the two jaw of the gripper. We also have another background task for the \textit{rgb} led present on media flange.

A correct thing to do before starting the use of the robot is to check and modify the safety configuration loaded by default. In our case:
\begin{itemize}
	\item for each joint in the robot configuration the values have been restricted by $ 2 $ degrees,
	\item moreover a protected workspace was added. In this case the robot never goes inside the constraint area.  
\end{itemize}

For further information about \textit{iiwa\_stack} you can see \cite{iiwa-stack-link} and the related work \cite{iiwa-stack-paper}. Instead for information about Sunrise OS and Sunrise Workbench you can see \cite{thesis}.
	
\section{Proposed teaching modalities}
\label{project-implementation}

As described in section~\ref{goals}, the main goal of the thesis is to evaluate, with an user study, different modalities to teach quickly and simply assembly tasks in industry. Staring from the goal, two modalities were evaluated:
\begin{enumerate}
	\item \kt{} teaching,
	\item \te{} teaching. 
\end{enumerate}

As described in the next sections, for each modality a way to save waypoints and actions on the gripper has been implemented. After that, the actions were captured by the program and saved in a \texttt{.csv} file. With a dedicated program and that file, all the actions saved into the file can be replicated by the robot. Below there's the description of the two developed modalities.

\subsection{\Kt{} teaching}
\label{teach}

As described in section \ref{robot-learning}, \kt{} teaching or also called \textit{teach by demonstration} is a way to move the robot in gravity compensation. Using the \textit{iiwa\_stack}, the gravity compensation mode has been implemented using joint impedance control mode where for every joint in robot configuration a stiffness and damping value has been set. Stiffness value must be grater than $ 0 $ and it's expressed in $ Nm/rad $, instead damping value must be between $ 0 $ and $ 1 $. After changing the control mode to joint impedance, the robot seems fall. A force contrary to the gravity must be carried out to keep it up.

On media flange, the green button (\circled{3} in figure \ref{media_flange}) was dedicated for saving commands. When the button is pressed according to the duration of the pressure, an action is recorded:
\begin{itemize}
	\item one click for save a waypoint,
	\item two seconds pressure for save the action on the gripper and the actual pose,
	\item five seconds pressure to close the teaching phase.
\end{itemize}
When an action is performed by the user, the led strip (\circled{1} in figure \ref{media_flange}) change color: green for waypoints, blue for gripper's actions.

\begin{figure}[H]
	\centering
	\includegraphics[scale=0.6]{media_flange.png}
	\caption[Media flange mounted on \kuka{}]{Media flange mounted on \kuka{}: \protect\circled{1} led strip, \\\protect\circled{2} enabling switch, \protect\circled{3} application button}
	\label{media_flange}
\end{figure}

\subsection{\Te{} teaching}
\label{teleoperation-pad}

\Te{}, or \textit{remote control}, indicates the movement of the robot at a distance. In our case, \te{} is intended as control of the \kuka{} with a \textbf{PlayStation 4 pad} and it was developed in two ways inspired from \cite{comparing-teleoperation}:

\begin{itemize}
	\item \textit{full task space}: in this modality the robot moves respect to the base frame. A movement along $ x $, $ y $ and $ z $ is performed using linear and angular velocity. With linear velocity you move the pose of the robot in task space, instead with angular velocity you move the orientation of the EE,
	\item \textit{reduced joint space}: in this other modality not all the joints are considered. Some joints have been blocked: you can move only the $ 1 $°, $ 2 $° and $ 6 $° joint. It's also possible to change the orientation of the EE using angular velocity.
\end{itemize}

In both cases, the buttons used are the same. In figure \ref{pad} we can see the structure of the pad, while now the buttons used are presented
\begin{itemize}
	\item $ \times $: close or open gripper, therefore this command automatically save the action and the actual pose,
	\item $ \triangle $: save the actual pose of the robot,
	\item $ \square $: change the control mode from position control to cartesian impedance control, and vice versa.
\end{itemize}

Using the pad, it's possible to take advantage of the internal haptic feedback. Therefore, when an external force greater than a preset value is detected, the pad start vibrating. Based on the control mode that the user is using (impedance or position control), the activation force is different.
Both of the modalities are performed using \texttt{L1} or \texttt{R1} as deadman buttons with the right or left analog. For our experiment only the first modality was used because it was considered simpler.\\

% todo immagine del pad
\begin{figure}[H]
	\centering
	\includegraphics[scale=0.25]{ps4_pad.jpg}
	\caption{PlayStation 4 pad}
	\label{pad}
\end{figure}
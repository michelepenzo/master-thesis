\documentclass{paper}

\usepackage[utf8]{inputenc}
\usepackage[italian]{babel}
\usepackage{a4wide}			% size page
\usepackage{hyperref}		% links
\usepackage{graphicx}		% imgs
\usepackage{float}			% due immagini
\usepackage{biblatex}		% bibliografia

% utils
\graphicspath{{figures/}} 	
\setlength\parindent{0pt}		% permanent no indent 

% biblio
\addbibresource{Bibliografia}

% new commands
\newcommand{\var}[1]{\texttt{#1}}

% title
\title{ Industrial Interface for KUKA Robot \\ 	\large{Robotica AA 2019-2020} }
   
% info
\author{Michele Penzo}
\date{\today}

\begin{document}
\maketitle 

\section{Introduzione}
	L'obiettivo del progetto è quello di lavorare, capire e trovare il miglior modo per interfacciarsi con il Robot KUKA IIWA LBR. 
	Possiamo dividere i task del progetto in:
	\begin{enumerate}
		\item valutare le soluzioni proposte
		\item creare un piccolo task pick and place in simulata
		\item eseguire il tutto sul robot reale
	\end{enumerate}
	Link repository progetto: \href{https://github.com/michelepenzo/robotica}{https://github.com/michelepenzo/robotica}

\section{Stack utilizzato: IFL-CAMP/iiwa\_stack}
	\label{ifl-camp}
	Questo tool mette a disposizione un nodo \textbf{RosJava} che viene eseguito sul robot come se fosse una Sunrise Application. Quindi fornisce topic/servizi ROS e funziona con i parametri classici di ROS.
	Ha il supporto a a \textbf{Gazebo} e \textbf{MoveIt}, cosi è possibile usare il robot in simulazione.
	Ha il supporto a ROS Melodic, anche se nella wiki dicono che sia possibile usarlo anche con Melodic.

\section{Descrizione del gripper}	
	Il gripper utilizzato è un \var{mplm3240}, composto da:
	\begin{itemize}
		\item flangia di supporto: in figura \ref{base};
		\item tool di supporto con i $ 2 $ giunti prismatici: in figura \ref{tool};
	\end{itemize}
	
	Il gripper completo è stato montato sul \textit{iiwa14} tenendo conto dell'offset:
	\begin{itemize}
		\item \var{x}: $ 0 $;
		\item \var{y}: $ 0 $;
		\item \var{z}: $ 118.7 $mm.
	\end{itemize} 


	\begin{figure}[H]
		\centering
		\begin{minipage}{.5\textwidth}
			\centering
			\includegraphics[width=.4\linewidth]{base.png}
			\caption{Flangia di supporto.}
			\label{base}
		\end{minipage}%
		\begin{minipage}{.5\textwidth}
			\centering
			\includegraphics[width=.4\linewidth]{tool.png}
			\caption{Gripper con 2 giunti prismatici.}
			\label{tool}
		\end{minipage}
	\end{figure}

\end{document}